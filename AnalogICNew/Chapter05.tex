\chapter{CMOS运算放大器·基础}
运算放大器是一个具有相当高增益的放大器。我们很早就知道,当开环增益足够大时,在负反馈下的闭环增益就仅取决于反馈结构,而与开环增益无关,利用这个原理可以设计许多重要的模拟电路。也因此,运算放大器可以说是模拟电路中最为重要且应用最为广泛的部件。

运算放大器的主要要求是具有足够大的增益(开环增益),然而,在\xref{chap:CMOS放大器}中介绍的所有单级放大器都没有足够大的增益,因此,运算放大器往往采用多级结构。我们在\xref{chap:CMOS放大器}伊始就曾提到,运算放大器由三级组成:输入级、中间级、输出级。但实际上输出级不总是必须的。之所以有时需要输出级,是因为通常单级放大器的输出电阻都很大,这可以驱动一个电容负载,但却不能很好的驱动一个电阻负载,那样大部分电压都会在放大器内部的输出电阻上浪费掉。

取决于输出电阻的大小,我们可以将运算放大器分为两类
\begin{itemize}
    \item 高输出电阻放大器,无输出级,用“无缓冲”描述,也称为运算跨导放大器(OTA)。
    \item 低输出电阻放大器,有输出级,用“有缓冲”描述,也称为电压运算放大器。
\end{itemize}
在本章,我们先研究的是无缓冲运算放大器,换言之,只需要输入级和中间级就可以了。最基础的无缓冲运算放大器称为“两级运算放大器”,它的输入级和中间级分别是\xref{chap:CMOS放大器}中我们已经非常熟悉的差分放大器和反相放大器,其性能不是特别好,但却是后续所有运放的基础。

在本章,我们将从两级运放开始,介绍运算放大器的补偿原理和电源抑制比的概念,并探讨运算放大器的设计方法,随后,再讨论如何通过共源共栅结构的应用,改善两级运放的性能。

% 它的输入级和中间级分别是\xref{chap:CMOS放大器}中我们已经非常熟悉的差分放大器和反相放大器。两级运放的性能并不是特别好,但它是后续所有运放的基础。

% 本章将首先

% 本章首先将从数学角度探讨二阶反馈系统的特性,运算放大器必然是在负反馈下工作的,然而负反馈会导致一个问题:相位裕度。该部分的数学分析将引出相位裕度的概念,并解释为什么不做补偿的运算放大器将具有很低的相位裕度。随后将介绍两级运放的结构,并讨论两级运放的补偿方法、设计方法、电源抑制比。最后将介绍如何应用共源共栅结构改进运放的特性。

% \section{反馈的数学模型}
\xref{fig:反馈系统的框图}展示了一个反馈系统的信号流图,其中,$A_v(s)$是开环增益,$F$是反馈系数,$A_{in}$是输入端的一个衰减系数,$A_{in}$反映的是实际电路中由于反馈结构的引入使得输入会经过一定的分压才能到达放大器的输入端,我们后面会看到$A_{in}$的存在对于正确建模反馈电路是必须的。

现在我们想要计算的是放大器的闭环增益$A_{vf}(s)=v_{out}(s)/v_{in}(s)$是多少。

\begin{Figure}[反馈系统的框图]
    \includegraphics[scale=0.8]{build/Chapter05A_02.fig.pdf}
\end{Figure}

首先,$v_{out}(s)$是$A_{v}(s)$放大$A_{in}v_{in}(s)$和$-Fv_{out}(s)$的结果
\begin{Equation}
    v_{out}(s)=A_v(s)[A_{in}v_{in}(s)-Fv_{out}(s)]
\end{Equation}
展开整理得到
\begin{Equation}
    v_{out}(s)=A_v(s)A_{in}v_{in}(s)-A_v(s)Fv_{out}(s)
\end{Equation}
将$v_{out}(s)$的项都移至左侧
\begin{Equation}
    v_{out}(s)[1+A_v(s)F]=v_{in}(s)A_v(s)A_{in}
\end{Equation}
即得
\begin{Equation}
    A_{vf}(s)=\frac{v_{out}(s)}{v_{in}(s)}=\frac{A_v(s)A_{in}}{1+A_v(s)F}
\end{Equation}
和之前习惯一致,在低频$s=0$下记$A_{vf}(s)=A_{vf}$且$A_v(s)=A_v$
\begin{Equation}
    A_{vf}=\frac{A_vA_{in}}{1+A_vF}
\end{Equation}
由于低频下运算放大器的增益$A_v$都是相当大的,因此可以适用深度负反馈近似$A_{v}F\gg 1$
\begin{Equation}
    A_{vf}=\frac{A_{in}}{F}
\end{Equation}
这里要解释一下为什么低频下才做深度负反馈近似。这是因为$A_{v}$一定很大但$A_{v}(s)$就不一定了,任何放大器的增益都会随着频率增加逐渐减小至单位增益,故$A_{v}(s)$下不应做该近似。

\begin{BoxFormula}[反馈的闭环增益]
    闭环增益可以表达为
    \begin{Equation}
        A_{vf}(s)=\frac{A_v(s)A_{in}}{1+A_v(s)F}
    \end{Equation}
    闭环增益在低频下可以适用深度负反馈近似
    \begin{Equation}
        A_{vf}=\frac{A_vA_{in}}{1+A_vF}=\frac{A_{in}}{F}
    \end{Equation}
\end{BoxFormula}
请注意!低频/高频、开环/闭环是两个维度,不要搞混$A_{v},A_{v}(s),A_{vf},A_{vf}(s)$之间的关系。

现在我们用如\xref{fig:同相放大器}和\xref{fig:反相放大器}所示的同相放大器和反相放大器来验证一下上述理论。
\begin{Figure}[典型的运放电路]
    \begin{FigureSub}[同相放大器]
        \includegraphics[scale=0.8]{build/Chapter05A_03.fig.pdf}
    \end{FigureSub}
    \hspace{0.25cm}
    \begin{FigureSub}[反相放大器]
        \includegraphics[scale=0.8]{build/Chapter05A_04.fig.pdf}
    \end{FigureSub}
\end{Figure}
1. 同相放大器的闭环增益是
\begin{Equation}
    A_{vf}=1+\frac{R_2}{R_1}
\end{Equation}
对于反馈系数$F$,输出$v_{out}$经$R_1,R_2$在$R_1$上的分压是反馈电压
\begin{Equation}
    F=\frac{R_1}{R_1+R_2}
\end{Equation}
对于输入系数$A_{in}$,输入$v_{in}$之间连接到了放大器,故为$1$
\begin{Equation}
    A_{in}=1
\end{Equation}
根据\xref{fml:反馈的闭环增益}验证结果
\begin{Equation}
    A_{vf}=\frac{A_{in}}{F}=\frac{R_1+R_2}{R_1}=1+\frac{R_2}{R_1}
\end{Equation}

2. 反相放大器的闭环增益是
\begin{Equation}
    A_{vf}=-\frac{R_2}{R_1}
\end{Equation}
对于反馈系数$F$,输出$v_{out}$经$R_1,R_2$在$R_1$上的分压是反馈电压
\begin{Equation}
    F=\frac{R_1}{R_1+R_2}
\end{Equation}
对于输入系数$A_{in}$,我们注意到,这里输入无法直接到达放大器了!真正到达放大器的输入电压是$v_{in}$经$R_1,R_2$在$R_2$上的分压,且要取一个负号,考虑到连接的是负输入端
\begin{Equation}
    A_{in}=-\frac{R_2}{R_1+R_2}
\end{Equation}
根据\xref{fml:反馈的闭环增益}验证结果
\begin{Equation}
    A_{vf}=\frac{A_{in}}{F}=-\frac{R_2}{R_1+R_2}\frac{R_1+R_2}{R_1}=-\frac{R_2}{R_1}
\end{Equation}
由此也可以看出为什么我们一定要有$A_{in}$项,否则反相放大器将无法适用反馈模型!

\section{二阶反馈系统的频域分析}\setpeq{二阶反馈系统的频域分析}
一阶系统的频域函数是
\begin{Equation}&[1]
    A_{v}(s)=\frac{A_v}{(1-s/\omega_{p1})}
\end{Equation}
二阶系统的频域函数是
\begin{Equation}&[2]
    A_{v}(s)=\frac{A_v}{(1-s/\omega_{p1})(1-s/\omega_{p2})}
\end{Equation}
二阶反馈系统本质上仍是二阶系统,当然也可以适用\xrefpeq{2}的表达式,但不同之处在于,通常的极点$\omega_{p1},\omega_{p2}$是一个负的实数,而反馈系统的极点$\omega_{p1},\omega_{p2}$可能具有虚部!因而我们愿意用一种略微不同的形式来表达二阶反馈系统的频域函数,以更好体现反馈将如何影响系统
\begin{Equation}&[3]
    A_{vf}(s)=\frac{A_{vf}}{1+2\zeta s/\omega_n+s^2/\omega_n^2}
\end{Equation}
这里$\zeta,\omega_n$是待说明含义的参量。我们现在要考虑这样一个问题,假设有一个如\xrefpeq{2}所示的开环的二阶系统,具有$\omega_{p1},\omega_{p2}$的极点和$A(s)$的开环增益,现在按\xref{fig:反馈系统的框图}施加反馈系数和输入系数分别为$F$和$A_{in}$的负反馈,得到的闭环增益$A_{vf}(s)$将是什么?或者说,得到的闭环增益$A_{vf}(s)$如果写成\xrefpeq{3}的形式,其中$\zeta,\omega_n$将是什么形式?新的极点$\omega_{p1}',\omega_{p2}$又会位于哪里?这并不复杂,我们可以从描述$A_{vf}(s)$与$A_v(s)$关系的\xref{fml:反馈的闭环增益}开始
\begin{Equation}&[4]
    A_{vf}(s)=\frac{A_v(s)A_{in}}{1+A_v(s)F}
\end{Equation}
这里的$A_v(s)$是一个二阶系统,代入\xrefpeq{2}
\begin{Equation}&[5]
    A_{vf}(s)=\frac{A_vA_{in}}{(1-s/\omega_{p1})(1-s/\omega_{p2})+A_vF}
\end{Equation}
展开得到
\begin{Equation}&[6]
    A_{vf}(s)=\frac{A_vA_{in}}{1+A_vF-s(\omega_{p1}+\omega_{p2})\omega_{p1}\omega_{p2}+s^2/\omega_{p1}\omega_{p2}}
\end{Equation}
上下同除$1+A_vF$
\begin{Equation}&[7]
    \qquad\qquad
    A_{vf}(s)=\frac{A_vA_{in}/(1+A_vF)}{1-s(\omega_{p1}+\omega_{p2})/[\omega_{p1}\omega_{p2}(1+A_vF)]+s^2/[\omega_{p1}\omega_{p2}(1+A_vF)]}
    \qquad\qquad
\end{Equation}
此时,参照\xref{fml:反馈的闭环增益},注意到\xrefpeq{7}的分子恰好是$A_{vf}$
\begin{Equation}&[8]
    \qquad\qquad
    A_{vf}(s)=\frac{A_{vf}}{1-s(\omega_{p1}+\omega_{p2})/[\omega_{p1}\omega_{p2}(1+A_vF)]+s^2/[\omega_{p1}\omega_{p2}(1+A_vF)]}
    \qquad\qquad
\end{Equation}
现在我们对比\xrefpeq{3}和\xrefpeq{8}
\begin{Equation}
    \frac{2\zeta}{\omega_n}=-\frac{\omega_{p1}+\omega_{p2}}{\omega_{p1}\omega_{p2}(1+A_vF)}\qquad \frac{1}{\omega_n^2}=\frac{1}{\omega_{p1}\omega_{p2}(1+A_vF)}
\end{Equation}
求出$\omega_n$
\begin{Equation}
    \omega_n=\sqrt{\omega_{p1}\omega_{p2}(1+A_vF)}
\end{Equation}
求出$\zeta$
\begin{Equation}
    \zeta=-\frac{1}{2}\frac{\omega_{p1}+\omega_{p2}}{\sqrt{\omega_{p1}\omega_{p2}(1+A_vF)}}
\end{Equation}
将结论整理如下
\begin{BoxFormula}[二阶反馈系统的频域特性]
    二阶反馈系统的频域特性为
    \begin{Equation}
        A_{vf}(s)=\frac{A_{vf}}{1+2\zeta s/\omega_n+s^2/\omega_n^2}
    \end{Equation}
    其中$\zeta$和$\omega_n$分别为
    \begin{Equation}
        \zeta=-\frac{1}{2}\frac{\omega_{p1}+\omega_{p2}}{\sqrt{\omega_{p1}\omega_{p2}(1+A_vF)}}\qquad
        \omega_n=\sqrt{\omega_{p1}\omega_{p2}(1+A_vF)}
    \end{Equation}
\end{BoxFormula}
现在让我们来分析一下结论,\xref{fig:二阶反馈系统的频域特性--频域特性}展示了不同$\zeta$下$|A_{vf}(s)|$随$s=\j\omega$变化的曲线,这里参数$\zeta$被称为阻尼系数,$\zeta<1$称为欠阻尼,$\zeta>1$称为过阻尼。我们注意到$|A_{vf}(s)|$随$\omega$的变化整体呈现一个低通特性,但是当$\zeta<\sqrt{2}/2=0.707$时在$\omega=\omega_n$处会出现一个尖峰。
\begin{Figure}[二阶反馈系统的频域特性]
    \begin{FigureSub}[频域特性;二阶反馈系统的频域特性--频域特性]
        \includegraphics[scale=0.8]{build/Chapter05A_01a.fig.pdf}
    \end{FigureSub}
    \begin{FigureSub}[频域特性的最大值;二阶反馈系统的频域特性--频域特性最大值]
        \includegraphics[scale=0.8]{build/Chapter05A_01b.fig.pdf}
    \end{FigureSub}
\end{Figure}
\xref{fig:二阶反馈系统的频域特性--频域特性最大值}展示了峰值随$\zeta$的变化(仅对$\zeta<0.707$有效),可以证明峰值符合以下公式
\begin{BoxFormula}[二阶反馈系统的频域峰值]
    二阶反馈系统的频域特性的峰值为
    \begin{Equation}
        \max|A_{vf}(s)|=\frac{1}{2\zeta\sqrt{1-\zeta^2}}
    \end{Equation}
\end{BoxFormula}


不过我们可能更关心的是新极点的位置!根据\xref{fml:二阶反馈系统的频域特性}中$A_{vf}(s)$表达式的分母
\begin{Equation}
    1+2\zeta s/\omega_n+s^2/\omega_n^2=0
\end{Equation}
两边同乘以$\omega_{n}$,整理得到
\begin{Equation}
    s^2+2\zeta\omega_ns+\omega_n^2=0
\end{Equation}
这里$s$的两个解分别记为$\omega_{p1}',\omega_{p2}'$,若假设$\zeta<1$,则产生一组共轭复根
\begin{Gather}
    \omega_{p1}'=-\omega_n\zeta+\j\omega_n\sqrt{1-\zeta^2}\\
    \omega_{p2}'=-\omega_n\zeta-\j\omega_n\sqrt{1-\zeta^2}
\end{Gather}

注意到$\omega_{p1}'$和$\omega_{p2}'$的模实际上就是$\omega_n$
\begin{Equation}
    |\omega_{p1}'|=|\omega_{p2}'|=\omega_n
\end{Equation}
注意到$\omega_{p1}'$和$\omega_{p2}'$的实部$-\omega_n\zeta$恰是$\omega_{p1},\omega_{p2}$的平均值(考虑\xref{fml:二阶反馈系统的频域特性}中$\zeta,\omega_n$的式子)
\begin{Equation}
    \Re\omega_{p1}'=\Re\omega_{p2}'=-\omega_n\zeta=\frac{\omega_{p1}+\omega_{p2}}{2}
\end{Equation}

\xref{fig:反馈对极点的影响}可视化了$\omega_{p1},\omega_{p2},\omega_{p1}',\omega_{p2}'$的位置,对于一个二阶反馈系统(在欠阻尼$\zeta<1$时)
\begin{itemize}
    \item 反馈会使极点变为一对共轭复数。
    \item 闭环极点$\omega_{p1}',\omega_{p2}'$的实部是开环极点$\omega_{p1},\omega_{p2}$的中间点。
    \item 闭环极点$\omega_{p1}',\omega_{p2}'$至原点的距离是$\omega_n$,这也是参量$\omega_n$的实际意义。
\end{itemize}
\begin{Figure}[反馈对极点的影响]
    \includegraphics[scale=0.8]{build/Chapter05A_05.fig.pdf}
\end{Figure}

最后我们再思考一个问题,即$\zeta,\omega_n$关于$\omega_{p1},\omega_{p2},A_v,F$的表达式到底说明了什么。

对于$\omega_n$
\begin{Equation}
    \omega_n=\sqrt{\omega_{p1}\omega_{p2}(1+A_vF)}
\end{Equation}
若将反馈系数$F$视为$F=1$\footnote{若$A_{in}=1$则$A_{vf}=1/F$,故$F=1$相当于$A_{vf}=1$的电压跟随器。},则$\omega_n$大约是$\omega_{p1},\omega_{p2}$几何平均的$\sqrt{A_v}$倍,换言之,在闭环下的$\omega_n$要比$|\omega_{p1}|,|\omega_{p2}|$大的多。我们知道,幅频特性中频率每遇到一个极点$|\omega_{p1}|,|\omega_{p2}|$就会增加$\SI{-20}{dB.dec^{-1}}$的下降速率,而对于极点是复数的情况,幅频特性中频率是在遇到极点的模,即$\omega_n$时开始下降。由此可见,反馈令放大器的带宽变大了,这是增益减小换来的。

对于$\zeta$
\begin{Equation}
    \zeta=-\frac{1}{2}\frac{\omega_{p1}+\omega_{p2}}{\sqrt{\omega_{p1}\omega_{p2}(1+A_vF)}}
\end{Equation}
假设$\omega_{p1}=\omega_{p2}$且$F=1$
\begin{Equation}
    \zeta=\frac{1}{\sqrt{A_v}}
\end{Equation}
假设$\omega_{p1}\ll\omega_{p2}$且$F=1$
\begin{Equation}
    \zeta=\frac{1}{2\sqrt{A_v}}\sqrt{\frac{\omega_{p2}}{\omega_{p1}}}
\end{Equation}
这就告诉我们,由于$A_v$很大,通常而言,阻尼系数$\zeta$是非常接近$0$的值,不过,如果能让两个极点$\omega_{p1},\omega_{p2}$相距比较远,阻尼系数$\zeta$亦会因正比于$\sqrt{\omega_{p2}/\omega_{p1}}$变得稍大一些。但总的来说,在模拟集成电路设计的背景下,阻尼系数$\zeta$都是位于$0<\zeta<1$的欠阻尼区间中的。

\section{二阶反馈系统的时域分析}\setpeq{二阶反馈系统的时域分析}
根据\xref{fml:二阶反馈系统的频域特性}
\begin{Equation}&[1]
    A_{vf}(s)=\frac{A_{vf}}{1+2\zeta s/\omega_n+s^2/\omega_n^2}
\end{Equation}
上下同乘$\omega_n^2$
\begin{Equation}&[2]
    A_{vf}(s)=\frac{A_{vf}\omega_n^2}{s^2+2\zeta\omega_ns+\omega_n^2}
\end{Equation}
我们现在想求出该系统的阶跃相应,换言之,要求出$A_{vf}(s)$的拉普拉斯逆变换对时间的积分。

我们知道有以下拉普拉斯逆变换关系
\begin{Equation}&[3]
    \frac{k}{(s-\alpha)^2+k^2}\to \e^{\alpha t}\sin kt
\end{Equation}
而其积分
\begin{Equation}&[4]
    \Int[0][t]\e^{\alpha\tau}\sin k\tau\dd{\tau}=\frac{1}{\alpha^2+k^2}\qty[k+\e^{\alpha t}(\alpha\sin kt-k\cos kt)]
\end{Equation}
应用辅助角公式
\begin{Equation}&[5]
    \Int[0][t]\e^{\alpha\tau}\sin k\tau\dd{\tau}=\frac{1}{\alpha^2+k^2}\qty[k+\e^{\alpha t}\sqrt{a^2+k^2}\sin(kt+\phi)]
\end{Equation}
其中$\phi$为
\begin{Equation}&[6]
    \phi=-\arctan(\frac{k}{\alpha})
\end{Equation}
将\xrefpeq{3}的分母展开
\begin{Equation}&[7]
    (s-\alpha)^2+k^2=s^2-2\alpha s+\alpha^2+k^2
\end{Equation}  
将\xrefpeq{7}与\xrefpeq{2}的分母比较
\begin{Equation}&[8]
    -2\alpha=2\zeta\omega_n\qquad \alpha^2+k^2=\omega_n^2
\end{Equation}
故有
\begin{Equation}&[9]
    \alpha=-\omega_n\zeta\qquad k=\omega_n\sqrt{1-\zeta^2}
\end{Equation}
这样一来,\xrefpeq{2}就可以表示为
\begin{Equation}&[10]
    A_{vf}(s)\frac{A_{vf}\omega_n}{\sqrt{1-\zeta^2}}\frac{k}{(s-\alpha)^2+k^2}
\end{Equation}
故$A_{vf}(s)$对应的时域阶跃响应$v_{OUT}(t)$是
\begin{Equation}&[11]
    v_{OUT}(t)=\frac{A_{vf}\omega_n}{\sqrt{1-\zeta^2}}\frac{1}{\alpha^2+k^2}\qty[k+\e^{\alpha t}\sqrt{\alpha^2+k^2}\sin(kt+\phi)]
\end{Equation}
代入\xrefpeq{9}给出的$\alpha$和$k$
\begin{Equation}&[12]
    \qquad\qquad
    v_{OUT}(t)=\frac{A_{vf}\omega_n}{\sqrt{1-\zeta^2}}\frac{1}{\omega_n^2}\qty[\omega_n\sqrt{1-\zeta^2}+\e^{-\omega_n\zeta t}\omega_n\sin(\omega_n\sqrt{1-\zeta}t+\phi)]
    \qquad\qquad
\end{Equation}
化简得到
\begin{Equation}
    v_{OUT}(t)=A_{vf}\qty[1+\frac{1}{\sqrt{1-\zeta^2}}\e^{-\omega_n\zeta t}\sin(\omega_n\sqrt{1-\zeta^2}t+\phi)]
\end{Equation}
其中$\phi$根据\xrefpeq{6}代入\xrefpeq{9}得到
\begin{Equation}
    \phi=-\arctan(\frac{k}{\alpha})=\arctan(\frac{\sqrt{1-\zeta^2}}{\zeta})
\end{Equation}
将结论整理如下
\begin{BoxFormula}[二阶反馈系统的时域特性]
    二阶反馈系统的时域阶跃响应为
    \begin{Equation}
        v_{OUT}(t)=A_{vf}\qty[1+\frac{1}{\sqrt{1-\zeta^2}}\e^{-\omega_n\zeta t}\sin(\omega_n\sqrt{1-\zeta^2}t+\phi)]
    \end{Equation}
    其中$\phi$为
    \begin{Equation}
        \phi=\arctan(\frac{\sqrt{1-\zeta^2}}{\zeta})
    \end{Equation}
\end{BoxFormula}
上述结论仅适合$\zeta<1$的欠阻尼情形,如前文所述,我们主要关心的是欠阻尼。对于$\zeta>1$的过阻尼情形,需要用另外一组拉普拉斯逆变换,其解将包含双曲正弦函数。或者,如果愿意将这里的$\sin$视为一个复变函数,那么\xref{fml:二阶反馈系统的时域特性}对于欠阻尼$\zeta<1$和过阻尼$\zeta>1$都是成立的。

我们可以证明过冲和过冲发生时间分别是
\begin{BoxFormula}[二阶反馈系统的过冲]
    二阶反馈系统的时域阶跃响应的过冲量为
    \begin{Equation}
        v_{OUT}(t_p)/A_{vf}-1=\exp(-\frac{-\pi\zeta}{\sqrt{1-\zeta^2}})
    \end{Equation}
    其中过冲时间$t_p$为
    \begin{Equation}
        t_p=\frac{\pi}{\omega_n\sqrt{1-\zeta^2}}
    \end{Equation}
\end{BoxFormula}

在\xref{fig:二阶反馈系统的时域特性--时域阶跃响应}中展示了不同$\zeta$下阶跃响应$v_{OUT}(t)$随时间$t$的变化,观察到$\zeta<1$时会发生过冲。请注意,时域冲击响应在$\zeta<1$时的过冲和频域特性在$\zeta<0.707$时会有额外峰值这两件事间没有必然联系,不必过度解读为何两者的$\zeta$的分界线不同。\xref{fig:二阶反馈系统的时域特性--时域阶跃响应的过冲时间}展示了过冲量随$\zeta$的变化,过冲的定义是“峰值$-$终值$/$终值”。\xref{fig:二阶反馈系统的时域特性--时域阶跃响应的过冲时间}展示了过冲发生时间$t_p$随$\zeta$的变化。
\begin{Figure}[二阶反馈系统的时域特性]
    \begin{FigureSub}[时域阶跃响应;二阶反馈系统的时域特性--时域阶跃响应]
        \includegraphics[scale=0.8]{build/Chapter05A_01c.fig.pdf}
    \end{FigureSub}
    \begin{FigureSub}[时域阶跃响应的过冲量;二阶反馈系统的时域特性--时域阶跃响应的过冲量]
        \includegraphics[scale=0.8]{build/Chapter05A_01e.fig.pdf}
    \end{FigureSub}\\ \vspace{0.25cm}
    \begin{FigureSub}[时域阶跃响应的过冲时间;二阶反馈系统的时域特性--时域阶跃响应的过冲时间]
        \includegraphics[scale=0.8]{build/Chapter05A_01d.fig.pdf}
    \end{FigureSub}
\end{Figure}
通过这一小节的研究,我们建立了过冲和$\zeta$的关系,这样一来,对于一个未知的二阶反馈系统,只要能测定其阶跃响应的过冲量,我们就可以反推出其阻尼系数$\zeta$是多少了!

\section{相位裕度}
相位裕度到底是什么?让我们回到\xref{fig:反馈系统的框图},这里$L_v(s)=A_v(s)F$称为环路增益,在低频下这当然是一个负反馈(即$-L_v(s)$是负的),然而,我们知道,当频率越过一个极点后,相位会减小$\pi/2$。对于具有两个极点的二阶反馈系统,$-L_v(s)$的相位最终会从$\pi$减小到$0$,这意味着系统会从负反馈转变为正反馈,若此时$|L_v(s)|\geq 1$,那么这种正反馈将会是自激的,这样的系统是不稳定的。因此,相位裕度的定义就应是:当$|L_v(s)|=1$时$-L_v(s)$尚余的相位!

相位裕度通常记为$\phi_m$,现在我们要求的就是$\phi_m$关于$\zeta$的表达式。

根据\xref{fml:反馈的闭环增益}\setpeq{相位裕度}
\begin{Equation}
    A_{vf}(s)=\frac{A_v(s)A_{in}}{1+A_{v}(s)F}
\end{Equation}
两边同乘$1+A_v(s)F$
\begin{Equation}
    A_{vf}(s)+A_{vf}(s)A_v(s)F=A_v(s)A_{in}
\end{Equation}
整理得到
\begin{Equation}
    A_{vf}(s)+A_{v(s)}F[A_{vf}(s)-A_{in}F^{-1}]=0
\end{Equation}
由此就得到了环路增益$L_v(s)$的表达式
\begin{Equation}
    L_v(s)=A_{v}(s)F=\frac{A_{vf}(s)}{A_{in}F^{-1}-A_{vf}(s)}
\end{Equation}
关于$A_{vf}(s)$代入\xref{fml:二阶反馈系统的频域特性}
\begin{Equation}
    L_v(s)=\frac{A_{vf}}{A_{in}F^{-1}(1-2\zeta s/\omega_n+s^2/\omega_n^2)-A_{vf}}
\end{Equation}
然而,再次依据\xref{fml:反馈的闭环增益},有$A_{vf}=A_{in}F^{-1}$,因此
\begin{Equation}
    L_v(s)=\frac{1}{-2\zeta s/\omega_n+s^2/\omega_n^2}
\end{Equation}
我们记令$|L_v(s)|=1$的$s=\j\omega$中的$\omega$为$\omega_c$,称为截止频率。在上式中代入$s=\j\omega_c$
\begin{Equation}&[1]
    L_v(\j\omega_c)=\frac{1}{-\j 2\zeta(\omega_c/\omega_n)-(\omega_c/\omega_n)^2}
\end{Equation}
由于
\begin{Equation}
    |L_v(\j\omega_c)|=1
\end{Equation}
因此
\begin{Equation}
    |-\j 2\zeta(\omega_c/\omega_n)-(\omega_c/\omega_n)^2|=1
\end{Equation}
即
\begin{Equation}
    \sqrt{(\omega_c/\omega_n)^4+4\zeta(\omega_c/\omega_n)^2}=1
\end{Equation}
平方得到
\begin{Equation}
    (\omega_c/\omega_n)^4+4\zeta(\omega_c/\omega_n)^2-1=0
\end{Equation}
解得
\begin{Equation}
    (\omega_c/\omega_n)^2=\frac{1}{2}\qty[-4\zeta^2\pm\sqrt{16\zeta^4+4}]
\end{Equation}
化简,另外这里显然是取正的
\begin{Equation}
    (\omega_c/\omega_n)^2=\sqrt{4\zeta^4+1}-2\zeta^2
\end{Equation}
故有
\begin{Equation}
    \omega_c=\omega_n\qty[\sqrt{4\zeta^4+1}-2\zeta^2]^{1/2}
\end{Equation}\goodbreak
将该结论整理如下
\begin{BoxFormula}[二阶反馈系统的截止频率]
    二阶反馈系统的截止频率为
    \begin{Equation}
        \omega_c=\omega_n\qty[\sqrt{4\zeta^4+1}-2\zeta^2]^{1/2}
    \end{Equation}
\end{BoxFormula}
\begin{Figure}[二阶系统的若干特性]
    \begin{FigureSub}[二阶系统的相位裕度]
        \includegraphics[scale=0.8]{build/Chapter05A_01h.fig.pdf}
    \end{FigureSub}
    \begin{FigureSub}[二阶系统的过冲]
        \includegraphics[scale=0.8]{build/Chapter05A_01f.fig.pdf}
    \end{FigureSub}\\ \vspace{0.25cm}
    \begin{FigureSub}[二阶系统的截止频率]
        \includegraphics[scale=0.8]{build/Chapter05A_01g.fig.pdf}
    \end{FigureSub}
\end{Figure}\setpeq{相位裕度}
按照相位裕度$\phi_m$的定义,它是$-L_v(s)$的相位,当$s=\j\omega_c$时
\begin{Equation}
    \phi_m=\arg[-L_v(\j\omega_c)]
\end{Equation}
根据\xrefpeq{1}
\begin{Equation}
    \phi_m=\arctan\qty(\frac{2\zeta}{\omega_c/\omega_n})
\end{Equation}
就$\omega_c$代入\xref{fml:二阶反馈系统的截止频率}
\begin{Equation}
    \phi_m=\arctan(\frac{2\zeta}{(\sqrt{4\zeta^4+1}-2\zeta^2)^{1/2}})
\end{Equation}
我们可以证明这等价于
\begin{Equation}
    \phi_m=\arccos(\sqrt{4\zeta^4+1}-2\zeta^2)
\end{Equation}
将该结论整理如下
\begin{BoxFormula}[二阶反馈系统的相位裕度]
    二阶反馈系统的相位裕度为
    \begin{Equation}
        \phi_m=\arccos(\sqrt{4\zeta^4+1}-2\zeta^2)
    \end{Equation}
\end{BoxFormula}
\xref{fig:二阶系统的相位裕度}、\xref{fig:二阶系统的过冲}、\xref{fig:二阶系统的截止频率}依次展示了相位裕度、过冲、截止频率
\begin{itemize}
    \item 表示过冲的\xref{fig:二阶系统的过冲}与先前\xref{fig:二阶反馈系统的时域特性--时域阶跃响应的过冲量}完全相同,只不过纵轴换成了对数轴。
    \item 相位裕度$\phi_m$通常被认为至少要达到$\pi/4$,最好能达到$\pi/6$。
    \item 相位裕度$\phi_m$随$\zeta$的增大而增大,过冲则随$\zeta$的增大而减小,两者的变化是相反的。
    % \item 相位裕度$\phi_m$可以通过测定阶跃响应的过冲间接确定!
    \item 截止频率$\omega_c$在$\zeta$接近零时基本等于$\omega_n$,随着$\zeta$的增大$\omega_c$会变得略微小于$\omega_n$。
\end{itemize}


这里一个重要的想法是:相位裕度可以通过测定阶跃响应的过冲间接确定!
\section{两级运算放大器}

% \subsection{两级运放的电路结构}
两级运算放大器是最简单的运算放大器,它的电路结构如\xref{fig:两级运算放大器}所示
\begin{itemize}
    \item 第一级是一个电流镜负载的差分放大器,参见\xref{subsec:电流镜负载差分放大器的大信号特性}。
    \item 第二级是一个电流源负载的反相放大器,参见\xref{subsec:电流源负载反相放大器的大信号特性}。
    \item 差分放大器包括$M_1,M_2,M_3,M_4,M_5$,值得注意的是,之前是用$M_1$作正输入端而$M_2$作负输入端,这时放大器具有正的差模增益。此处恰好反过来了,故差分放大器的增益和后一级的反相放大器的增益就都是负的了,这样一来,整个运放的增益就是正的。
    \item 反相放大器包括$M_6,M_7$,有些不一样的是,这里是用PMOS的$M_6$作为放大管,而作为NMOS的$M_7$却是负载管。这样设计可能是为了让$M_7$和$M_5$能共用同一偏置。
    \item 采用正电源$V_{DD}$和负电源$V_{SS}$供电(而不是$V_{DD}$和GND)。
\end{itemize}

应指出的是,这里“两级运放”中的“两级”不应当视为一种对级数的泛指(事实上,大部分无缓冲的运算放大器都是两级的)。“两级运放”的名称特指这种最基本的无缓冲运放结构。

\begin{Figure}[两级运算放大器]
    \includegraphics[scale=0.8]{build/Chapter05B_02.fig.pdf}    
\end{Figure}

% 没错!运算放大器并没有那么复杂,不是吗?完全是由我们最熟悉的电路组合而成的。


% \subsection{两级运放的频率特性}
由于构成两级运算放大器的差分放大器和反相放大器我们都已经分析的相当透彻了,这里小信号电路不需要从晶体管画起,如\xref{fig:两级运算放大器的小信号电路}所示,我们可以将每一级放大器视为一个具有一定的跨导$G_{m}$、输出电阻$R$、输出电容$C$的黑盒。这里要说明两点,第一,尽管研究单级放大器的时候,更常用的小信号参数是增益$A_v$和输出电阻$R_{out}$,但是用跨导$G_m$和输出电阻$R_{out}$表示其实可以更好凸显“输入电压先被放大管转换为电流再在负载管上转换回电压输出”的直观分析思路,且跨导$G_m$的表达式可以很容易的从$A_v=-G_m R_{out}$截取到。第二,这里的建模只考虑了每一级放大器输出节点处的电容,忽略了其他内部电容。这是个合理的近似,因为参考之前的经验,在分析频率特性时我们会考虑想当多的电容,但是,到了真正设计的时候我们只会考虑输出电容,因为负载电容远大于寄生电容,只有输出节点的电容是较重要的。
\begin{Figure}[两级运放的小信号电路]
    \includegraphics[scale=0.8]{build/Chapter05B_03.fig.pdf}
\end{Figure}

这里$G_{m1},R_{1},G_{m2},R_{2}$的表达式参照\xref{fml:电流镜负载差分放大器--右侧输出--差模增益}、\xref{fml:电流镜负载差分放大器--右侧输出--输出电阻}、\xref{fml:电流源负载反相放大器--电压增益}、\xref{fml:电流源负载反相放大器--输出电阻}确定为
\begin{BoxFormula}[两级运放--等效跨导和输出电阻]
    两级运放,两级放大器的等效跨导和输出电阻为
    \begin{Gather}
        G_{m1}=g_{m2}\\
        G_{m2}=g_{m6}\\
        R_1=(g_{ds2}+g_{ds4})^{-1}\\
        R_2=(g_{ds6}+g_{ds7})^{-1}
    \end{Gather}
\end{BoxFormula}

现在求$A_v(s)=v_{out}/v_{in}$,就\xref{fig:两级运放的小信号电路}的$v_{c},v_{out}$节点列出方程
\begin{Gather}
    G_{m1}v_{in}+sC_1v_v+R_1^{-1}v_c=0\\
    G_{m2}v_c+sC_2v_{out}+R_2^{-1}v_{out}=0
\end{Gather}
我们可以解得
\begin{Equation}
    A_{v}(s)=\frac{G_{m1}G_{m2}R_1R_2}{(1+sR_1C_1)(1+sR_2C_2)}
\end{Equation}
这个结果是显然的:运算放大器的增益$A_v(s)$是组成其的两个单级放大器的增益之积。同时注意到,两级运算放大器具有两个极点$\omega_{p1},\omega_{p2}$,分别由第一级和第二级的输出节点产生。

% 第一级放大器和第二级放大器的输出节点处的$R_1,C_1$和$R_2,C_2$分别提供了一个极点。

% 这里分子是低频增益$A_v=G_{m1}G_{m2}R_1R_2$,从分母上很容易判读出两个极点
% \begin{Equation}
%     \omega_{p1}=-R_1^{-1}C_1^{-1}\qquad \omega_{p2}=-R_2^{-1}C_2^{-1}
% \end{Equation}

\begin{BoxFormula}[两级运放--增益]
    两级运放,低频增益为
    \begin{Equation}
        A_v=G_{m1}G_{m2}R_1R_2
    \end{Equation}
\end{BoxFormula}

\begin{BoxFormula}[两级运放--零极点]
    两级运放,零极点为
    \begin{Gather}
        \omega_{p1}=-R_1^{-1}C_1^{-1}\\
        \omega_{p2}=-R_2^{-1}C_2^{-1}
    \end{Gather}
\end{BoxFormula}
一切看起来都很简单,不是吗?从结构上看,运算放大器不过就是两个单级放大器的级联,从性质上看,运算放大器包括增益和零极点在内的各种特性似乎都可以由构成其的单级放大器得到。那关于运算放大器到底还有什么新东西要研究?请看\xref{sec:两级运算放大器的补偿原理}有关“补偿”的讨论。

% 实际上,运算放大器真正困难的地方不在这里,
\section{两级运算放大器的补偿原理}

\subsection{相位裕度}
运算放大器和过去单级放大器最大的不同之处在于其需要在深度负反馈的条件下使用,这一特殊使用背景是运算放大器真正的困难所在!因为反馈会导致一些我们未曾考虑的新问题。

\xref{fig:反馈的框图}展示了一个负反馈系统,$A_v(s)$是运放的开环增益,$F$是反馈系数,$A_{in}$是反馈结构在输入附带造成的衰减,如果对$A_{in}$感到困惑,可以简单的认为$A_{in}=1$。现在我们考虑这样一个问题,这里的环路增益(输入经过放大和反馈回路再回到输入时的增益)为$A_v(s)F$,考虑到叠加时的负号,反馈的相位是$\pi+\arg A_v(s)F$,通常而言反馈系数$F$为正,则反馈相位就是$\pi+\arg A_v(s)$。这里的两级运放具有两个极点,我们知道,频率每经过一个极点相位就会产生$-\pi/2$的相移,因而,频率较高时$A_v(s)$就会具有$-\pi$的相移,这对于开环使用并没有什么影响,但在闭环下,这就意味着,随着频率的增加反馈的相位将从$\pi$逐渐减小至$0$,换言之,反馈的类型会从负反馈变为正反馈!倘若反馈的相位降低至$0$时反馈的模值$|A_v(s)F|$还高于$1$,这意味着这种正反馈是自激的(输出会在高频下振荡),这样的系统是不稳定的。

\begin{Figure}[反馈的框图]
    \includegraphics[scale=0.8]{build/Chapter05A_02.fig.pdf}
\end{Figure}

% 相位裕度$\phi_m$是度量稳定性的一个量。

从上述讨论中看出,由负反馈转变为正反馈是不可避免的,只要保证反馈的模值$|A_v(s)F|$在相位归零前先行低于单位增益就可以了,因为导致不稳定的并不是正反馈而是存在自激的正反馈。对于运算放大器的设计,我们并不清楚反馈系数$F$会在使用时被设置为多少,因此,我们要考虑$F=1$最坏情况(此时输出完全被反馈至输入),故要考察的模值$|A_v(s)F|=|A_v(s)|$。

相位裕度$\phi_m$是度量稳定性的一个量。相位裕度$\phi_m$的定义就是:当模值$|A_v(s)|$降低到单位增益时相位$\pi+\arg A_v(s)$还剩余多少?我们将单位增益处$s=\j\omega$的频率$\omega$称为单位增益带宽,记作$\te{GB}$。因此,总结起来说,相位裕度$\phi_m$就是单位增益带宽$\te{GB}$处反馈的相位!

相位裕度$\phi_m$要多少才比较合适呢?较常用的是下面两个值
\begin{itemize}
    \item $\phi_m=\pi/4$认为是保持稳定的最小相位裕度。
    \item $\phi_m=\pi/3$认为是一个较合适的相位裕度。
\end{itemize}
有关二阶反馈系统和相位裕度的更多讨论,参见\xref{ap:二阶反馈系统}的内容。

% 上述提到的单位增益带宽可以

现在我们考虑一个问题,两级运放的相位裕度大概有多少?我们可以做一个估计,如\xref{fig:补偿前两级运放的频率响应}所示,对于两级运算放大器,典型情况下其极点$\omega_{p1},\omega_{p2}$都远离原点且相互之间比较接近,不妨假设$\omega_{p2}$比$\omega_{p1}$高两个数量级,同时,假设运放的开环增益$A_v$是$\num{1e4}=\SI{80}{dB}$。

复习一下极点对幅频和相频的近似影响有助于我们下面的讨论
\begin{itemize}
    \item 幅频特性在每经过一个极点后,增加$\SI{-20}{dB.dec^{-1}}$的下降速率。
    \item 相频特性在每个极点前后$\SI{1}{dec}$的范围内产生$-\pi/2$的相移,在极点处为$-\pi/4$的相移。
\end{itemize}
现在回到\xref{fig:补偿前两级运放的频率响应},总共$\SI{80}{dB}$的增益,在$|\omega_{p1}|$和$|\omega_{p2}|$之间的$\SI{2}{dec}$以$\SI{-20}{dB.dec^{-1}}$的速率衰减掉一半,在$|\omega_{p2}|$后$\SI{1}{dec}$内以$\SI{-40}{dB.dec^{-1}}$的速率衰减至$\SI{0}{dB}$,换言之,这里单位增益带宽$\te{GB}$位于$|\omega_{p2}|$后的一个数量级处,而此时,相位由于频率经过$|\omega_{p1}|$和$|\omega_{p2}|$且已到达$|\omega_{p2}|$后一个数量级,刚好由$\pi$减至零。这意味着,两级运放的相位裕度$\phi_m=0$!这是极不好的,相位裕度为零代表不稳定。当然上述只是一个估测,若增益更小些或极点间的距离更远一些,相位裕度可以略大于零,但总的来说,无补偿的两级运放的相位裕度是远远不够的。
\begin{Figure}[补偿前两级运放的频率响应]
    \includegraphics[scale=0.8]{build/Chapter05B_06.fig.pdf}
\end{Figure}
从上述分析过程中可见,增益一定时,假若我们能通过某种方式增大两个$\omega_{p1},\omega_{p2}$极点间的距离,令$\te{GB}$出现在$|\omega_{p2}|$之前,那就能获得更大的相位裕度。这就是下一小节要讨论的补偿。

\subsection{米勒补偿的概念}
补偿就是指通过一些方式增大运放的相位裕度$\phi_m$,最基本的补偿方式就是米勒补偿,对于两级运算放大器,米勒补偿将在第二级的输入和输出间增加一个$C_c$电容,如\xref{fig:使用米勒补偿的两级运放的小信号电路}所示。
% 米勒补偿中“米勒”就是指这种补偿方式基于米勒定理。依照米勒定理,跨接电容$C_c$会在中间节点产生一个$G_{m2}R_2C_c$的米勒等效电容,使主极点$\omega_{p1}$减小,增大了极点间距离,按\xref{subsec:相位裕度}末的讨论这有利于相位裕度。不过,实际上$C_c$电容对$\omega_{p1},\omega_{p2}$都有影响,还会引入一个新的零点$\omega_{z1}$,因此在上述近似分析后我们还是有必要列方程仔细计算一下米勒补偿的影响。

列出$v_c$和$v_{out}$处的方程
\begin{Gather}
    G_{m1}v_{in}+sC_1v_c+R_1^{-1}v_c+sC_c(v_c-v_{out})=0\\
    G_{m2}v_c+sC_2v_{out}+R_2^{-1}v_{out}+sC_c(v_{out}-v_c)=0
\end{Gather}
应用Mathematica解得
\begin{Equation}
    A_v(s)=\frac{G_{m1}G_{m2}R_1R_2[1-sG_{m2}^{-1}C_c]}{1+a_1s+a_2s^2}
\end{Equation}
其中$a_1,a_2$分别是
\begin{Gather}
    a_1=R_1(C_1+C_c)+R_2(C_2+C_c)+G_{m2}R_1R_2C_c\\
    a_2=R_1R_2[C_1C_2+C_1C_c+C_2C_c]
\end{Gather}
对于$a_1$,很明显最后一项远大于前两项,因为其包含一个增益$G_{m2}R_2$
\begin{Equation}
    a_1=G_{m2}R_1R_2C_c
\end{Equation}
对于$a_2$,假定$C_c,C_2\gg C_1$,则可以只保留$C_2C_c$项
\begin{Equation}
    a_2=R_1R_2C_2C_c
\end{Equation}
极点$\omega_{p1},\omega_{p2}$为
\begin{Gather}
    \omega_{p1}=-\frac{1}{a_1}=-G_{m2}^{-1}R_1^{-1}R_2^{-1}C_c^{-1}\\
    \omega_{p2}=-\frac{a_1}{a_2}=-G_{m2}C_2^{-1}
\end{Gather}
零点$\omega_{z1}$为
\begin{Equation}
    \omega_{z1}=G_{m2}C_c^{-1}
\end{Equation}

\begin{Figure}[使用米勒补偿的两级运放的小信号电路]
    \includegraphics[scale=0.8]{build/Chapter05B_04.fig.pdf}
\end{Figure}

将零极点整理如下
\begin{BoxFormula}[两级运放--米勒补偿--零极点]
    两级运放,使用米勒补偿,零极点为
    \begin{Gather}
        \omega_{p1}=-G_{m2}^{-1}R_1^{-1}R_2^{-1}C_c^{-1}\\
        \omega_{p2}=-G_{m2}C_2^{-1}\\
        \omega_{z1}=G_{m2}C_c^{-1}
    \end{Gather}
\end{BoxFormula}
我们可以解读一下结果
\begin{itemize}
    \item 极点$\omega_{p1}$由$\omega_{p1}=-R_1^{-1}C_1^{-1}$变化到$-G_{m2}^{-1}R_1^{-1}R_2^{-1}C_c^{-1}$,相当于第一级放大器的输出极点处的电容由$C_1$变为了$G_{m2}R_2C_c$的米勒电容,显然,米勒补偿后$\omega_{p1}$显著减小了。
    \item 极点$\omega_{p2}$由$\omega_{p2}=-R_2^{-1}C_2^{-1}$变化到$\omega_{p2}=-G_{m2}C_2^{-1}$,根据\xref{fml:两级运放--等效跨导和输出电阻}的结论,我们知道$R_2^{-1}=g_{ds6}+g_{ds7}$而$G_{m2}=g_{m6}$,依据$g_m\gg g_{ds}$的关系,米勒补偿后$\omega_{p2}$事实上也增大了。这一结果可以这样直观理解,在高频下,跨接电容$C_c$可以视为短接的,这样一来当计算第二级的输出电阻(即$v_c=0$)时,原本$M_6,M_7$都相当于是电流源,现在$M_6$由于$C_c$短路变成了二极管,故第二级输出电阻就从$(g_{ds6}+g_{ds7})^{-1}$变为了$g_{m6}^{-1}$即$G_{m2}^{-1}$。
    \item 零点$\omega_{z1}=G_{m2}C_c^{-1}$是新增的。我们在\xref{subsec:直观分析--频率特性}就曾提及跨接电容会导致零点,这里可以给出一个定性的解释:从$v_{c}$至$v_{out}$原先存在一个经$M_6,M_7$反相放大的通路,而跨接电容$C_c$引入后,使$v_c$至$v_{out}$在高频下能经电容$C_c$导通,这是正相的。在某个特别的频率下,经$C_c$的正相通路和经$M_6,M_7$的反相通路会在输出相互抵消,这就是零点!
\end{itemize}

总的来说,米勒补偿使$\omega_{p1}$更小使$\omega_{p2}$更大,按照\xref{subsec:相位裕度}末尾的观点,增大$\omega_{p1},\omega_{p2}$间的距离有利于更大的相位裕度,因为$\te{GB}$会相对$\omega_{p2}$前移从而使得频率达到$\te{GB}$时$\omega_{p2}$产生的相移比原来更小。这也就是米勒补偿实现“补偿”的原理。同时,米勒补偿还会引入一个不期望零点$\omega_{z1}$,因为$\omega_{z1}$作为一个右半平面零点与左半平面极点一样都会令相位产生$-\pi/2$的相移,这对相位裕度不利。然而,只要$\omega_{z1}$远大于我们关心的频率,它就基本不会产生影响。

\xref{fig:补偿后两级运放的频率响应}直观展示了米勒补偿的影响,忽略$\omega_{z1}$的影响,假设米勒补偿使极点$\omega_{p1},\omega_{p2}$分别比原来减小和增加了$\SI{1}{dec}$,这使得$\te{GB}$恰好落在了$|\omega_{p2}|$处,从而获得了$\phi_m=\pi/4$的相位裕度。
\begin{Figure}[补偿后两级运放的频率响应]
    \includegraphics[scale=0.8]{build/Chapter05B_07.fig.pdf}
\end{Figure}

至此我们已经定性理解了米勒补偿的影响,现在进行定量的分析,我们要回答这样一个问题,若要实现特定的相位裕度,如$\phi_{m}=\pi/4$或$\phi_m=\pi/3$,需要满足什么条件?为此,首先要确定单位增益带宽$\te{GB}$的表达式,假若$|\omega_{p2}|\geq\te{GB}$那么$\te{GB}$的形式就比较简单,有
\begin{Equation}
    \te{GB}=A_v\cdot|\omega_{p1}|
\end{Equation}
根据\xref{fml:两级运放--增益}和\xref{fml:两级运放--米勒补偿--零极点}
\begin{Equation}
    \te{GB}=G_{m1}G_{m2}R_1R_2\cdot G_{m2}^{-1}R_1^{-1}R_2^{-1}C_c^{-1}
\end{Equation}
化简得到
\begin{Equation}
    \te{GB}=G_{m1}C_c^{-1}
\end{Equation}
\begin{BoxFormula}[两级运放--米勒补偿--单位增益带宽]
    两级运放,使用米勒补偿,单位增益带宽为
    \begin{Equation}
        \te{GB}=G_{m1}C_c^{-1}
    \end{Equation}
\end{BoxFormula}

相位和频率间的数学关系是由反正切函数表示的,具体而言
\begin{Equation}
    \qquad\qquad\quad
    \arg A_v(s)=-\arctan(\frac{\omega}{|\omega_{p1}|})-\arctan(\frac{\omega}{|\omega_{p2}|})-\arctan(\frac{\omega}{|\omega_{z1}|})
    \qquad\qquad\quad
\end{Equation}
相位裕度$\phi_m=\pi+\arg A_v(s)$且取$\omega=\te{GB}$
\begin{Equation}
    \phi_m=\pi-\arctan(\frac{\te{GB}}{|\omega_{p1}|})-\arctan(\frac{\te{GB}}{|\omega_{p2}|})-\arctan(\frac{\te{GB}}{|\omega_{z1}|})
\end{Equation}
由于$\te{GB}/|\omega_{p1}|=A_v$且$A_v$很大,故$\arctan(\te{GB}/|\omega_{p1}|)=\pi/2$,因此
\begin{Equation}
    \phi_m=\pi/2-\arctan(\frac{\te{GB}}{|\omega_{p2}|})-\arctan(\frac{\te{GB}}{|\omega_{z1}|})
\end{Equation}
我们不妨用$k_{p2},k_{z1}$分别表示$|\omega_{p2}|,|\omega_{z1}|$相对$\te{GB}$的倍数
\begin{Equation}
    |\omega_{p2}|=k_{p2}\te{GB}\qquad
    |\omega_{z1}|=k_{z1}\te{GB}
\end{Equation}
我们知道零点$\omega_{z1}$的存在对相位裕度是不利的,为了减轻其影响,我们至少令$|\omega_{z1}|$比$\te{GB}$高一个数量级,即$|\omega_{z1}|=10\te{GB}$或$k_{z1}=10$。在此基础上,对于一个期望的相位裕度$\phi_m$,我们都可以找到一个所需的$k_{p2}$使之成立,特别的,对于最关心的$\phi_m=\pi/4$和$\phi_m=\pi/3$
\begin{BoxFormula}[两级运放--米勒补偿--相位裕度和极点频率]
    若要求$\phi_m=\pi/4$,对于$k_{z1}=10$,则应有
    \begin{Equation}
        |\omega_{p2}|=1.22\te{GB}\qquad k_{p2}=1.22
    \end{Equation}
    若要求$\phi_m=\pi/3$,对于$k_{z1}=10$,则应有
    \begin{Equation}
        |\omega_{p2}|=2.22\te{GB}\qquad k_{p2}=2.22
    \end{Equation}
\end{BoxFormula}
在\xref{fig:在米勒补偿下相位裕度和极点的关系}中,绘制了$k_{p2}$关于$\phi_m$的曲线,其中$k_{z1}$取定为$k_{z1}=10$。

% 这里有一点要说明,前面我们在讨论\xref{fig:补偿前两级运放的频率响应}和\xref{fig:补偿后两级运放的频率响应}时认为极点和零点对相位的影响局限在其前后一个数量级的频率内,那为何这里已经令$|\omega_{z1}|=10\te{GB}$高出一个数量级了,还是会对相位有影响。具体而言,为何$\phi_m=\pi/4$的是$|\omega_{p2}|=1.22\te{GB}$而不是$|\omega_{p2}|=1.00\te{GB}$?这是因为零极点对相位的影响局限在前后一个数量级只是一种近似,准确的关系要用$\arctan$函数描述,超出一个数量级

现在我们已经知道在$k_{z1}=10$下对于一定的相位裕度$\phi_m$需要令$k_{p2}$取多少了,接下来,我们想知道这样一件事,跨接电容$C_c$要取什么样的值,可以保证一个特定的$k_{p2}$值的出现?

\begin{Figure}[在米勒补偿下相位裕度和极点的关系]
    \includegraphics[scale=0.8]{build/Chapter05B_01a.fig.pdf}
\end{Figure}

根据$|\omega_{p2}|=k_{p2}\te{GB}$,代入\xref{fml:两级运放--米勒补偿--零极点}和\xref{fml:两级运放--米勒补偿--单位增益带宽}\setpeq{米勒补偿下相位裕度和极点的关系}
\begin{Equation}&[1]
    G_{m2}C_2^{-1}=k_{p2}G_{m1}C_c^{-1}
\end{Equation}
求出$C_c$
\begin{Equation}&[2]
    C_c=k_{p2}G_{m1}G_{m2}^{-1}C_2
\end{Equation}
根据$|\omega_{z1}|=k_{z1}\te{GB}$,代入\xref{fml:两级运放--米勒补偿--零极点}和\xref{fml:两级运放--米勒补偿--单位增益带宽}
\begin{Equation}&[3]
    G_{m2}C_c^{-1}=k_{z1}G_{m1}C_c^{-1}
\end{Equation}
注意到$C_c$可以被约掉
\begin{Equation}&[4]
    G_{m2}=k_{z1}G_{m1}
\end{Equation}
这意味着,若要满足$|\omega_{z1}|=k_{z1}\te{GB}$的约束,两级运放中第一级和第二级的跨导$G_{m1},G_{m2}$不是完全自由的!后者需要是前者的$k_{z1}$即$10$倍。如果将$k_{z1}=G_{m1}^{-1}G_{m2}$代入\xrefpeq{1}中
\begin{Equation}
    C_c=k_{p2}k_{z1}^{-1}C_2
\end{Equation}
我们将结论整理如下
\begin{BoxFormula}[两级运放--米勒补偿--跨接电容]
    若要保证$|\omega_{p2}|=k_{p2}\te{GB}$,则跨接电容$C_c$需要满足
    \begin{Equation}
        C_c=k_{p2}G_{m1}G_{m2}^{-1}C_2
    \end{Equation}
    若要保证$|\omega_{z1}|=k_{z1}\te{GB}$,则两级跨导$G_{m1},G_{m2}$需要满足关系
    \begin{Equation}
        G_{m2}=k_{z1}G_{m1}
    \end{Equation}
\end{BoxFormula}

\subsection{米勒补偿和调零电阻}
通过\xref{subsec:米勒补偿的概念},我们看到零点的引入不利于相位裕度,在这一小节,我们试图在米勒补偿的基础上找到一种方法,消除零点的影响。如\xref{fig:使用米勒补偿和调零电阻的两级运放的小信号电路},在跨接电容$C_c$上串联一个调零电阻$R_z$。

\begin{Figure}[使用米勒补偿和调零电阻的两级运放的小信号电路]
    \includegraphics[scale=0.8]{build/Chapter05B_05.fig.pdf}
\end{Figure}

这里$C_c$与$R_z$串联的导纳是
\begin{Equation}
    sC_c\parallel R_z^{-1}=sC_c(1+sC_cR_z)^{-1}
\end{Equation}
列出$v_c$和$v_{out}$处的方程
\begin{Gather}
    G_{m1}v_{in}+sC_1v_c+R_1^{-1}v_c+sC_c(1+sC_cR_z)^{-1}(v_c-v_{out})=0\\
    \qquad\qquad\qquad G_{m2}v_c+sC_2v_{out}+R_2^{-1}v_{out}+sC_c(1+sC_cR_z)^{-1}(v_{out}-v_c)=0\qquad\qquad\qquad
\end{Gather}
应用Mathematica解得
\begin{Equation}
    A_v(s)=\frac{G_{m1}G_{m2}R_1R_2[1-s(G_{m2}^{-1}C_c-R_zC_c)]}{1+a_1s+a_2s^2+a_3s^3}
\end{Equation}
其中$a_1,a_2,a_3$分别是
\begin{Gather}
    a_1=R_1(C_1+C_c)+R_2(C_2+C_c)+R_2C_c+G_{m2}R_1R_2R_c\\
    a_2=R_1R_2(C_1C_2+C_1C_c+C_2C_c)+R_zC_c(R_1C_1+R_2C_2)\\
    a_3=R_1R_2R_zC_1C_2C_c
\end{Gather}
这里的分母出现了三次项!但没有关系,我们仍然可以使用类似的近似方法
\begin{Equation}
    \omega_{p1}=-\frac{1}{a_1}\qquad
    \omega_{p2}=-\frac{a_1}{a_2}\qquad
    \omega_{p3}=-\frac{a_2}{a_3}
\end{Equation}
对于$a_1$,和之前一样,最后一项远大于其他项
\begin{Equation}
    a_1=G_{m2}R_1R_2C_c
\end{Equation}
对于$a_2$,仍然可以近似为
\begin{Equation}
    a_2=R_1R_2C_2C_c
\end{Equation}
对于$a_3$,没什么可近似的
\begin{Equation}
    a_3=R_1R_2R_zC_1C_2C_c
\end{Equation}
极点$\omega_{p1},\omega_{p2},\omega_{p3}$为
\begin{Gather}
    \omega_{p1}=-\frac{1}{a_1}=-G_{m2}^{-1}R_1^{-1}R_2^{-1}C_c^{-1}\\
    \omega_{p2}=-\frac{a_1}{a_2}=-G_{m2}C_2^{-1}\\
    \omega_{p3}=-\frac{a_2}{a_3}=-R_z^{-1}C_1^{-1}
\end{Gather}
零点$\omega_{z1}$为
\begin{Equation}
    \omega_{z1}=(G_{m2}^{-1}C_c-R_zC_c)^{-1}
\end{Equation}
由此可见,调零电阻$R_z$使$\omega_{z1}=(G_{m2}^{-1}C_c)^{-1}$变为了$\omega_{z1}=(G_{m2}^{-1}C_c-R_zC_c)^{-1}$,这就使得通过调整$R_z$的大小可以影响零点的位置。除此之外,还引入了一个新极点$\omega_{p3}=-R_z^{-1}C_1^{-1}$。
\begin{BoxFormula}[两级运放--调零电阻--零极点]
    两级运放,使用带有调零电阻的米勒补偿,零极点为
    \begin{Gather}
        \omega_{p1}=-G_{m2}^{-1}R_1^{-1}R_2^{-1}C_c^{-1}\\
        \omega_{p2}=-G_{m2}C_2^{-1}\\
        \omega_{p3}=-R_z^{-1}C_1^{-1}\\
        \omega_{z1}=(G_{m2}^{-1}C_c-R_zC_c)^{-1}
    \end{Gather}
\end{BoxFormula}

现在我们开始调零,关于$\omega_{z1}$的处理,我们有两种思路
\begin{enumerate}
    \item 令$\omega_{z1}$变为无穷大,从而使该零点完全不影响频率特性。
    \item 令$\omega_{z1}=\omega_{p2}$,即将$\omega_{z1}$从一个右半平面零点变为一个左半平面零点,且令其恰好位于极点$\omega_{p2}$处。这样做是更好的,因为左半平面零点是可以增加相位的!将$\omega_{z1}$置于$\omega_{p2}$处可以使后者对相位的影响被抵消,这样一来对相位的有影响的就只有$\omega_{p1}$和$\omega_{p3}$了。
\end{enumerate}
对于第一种情况,根据\xref{fml:两级运放--调零电阻--零极点}
\begin{Equation}
    (G_{m2}^{-1}C_c-R_zC_c)^{-1}=0
\end{Equation}
很容易解得
\begin{Equation}
    R_z=G_{m2}^{-1}
\end{Equation}
对于第二种情况,根据\xref{fml:两级运放--调零电阻--零极点}
\begin{Equation}
    (G_{m2}^{-1}C_c-R_zC_c)^{-1}=-G_{m2}C_2^{-1}
\end{Equation}
两边一起取倒数
\begin{Equation}
    G_{m2}^{-1}C_c-R_zC_c=-G_{m2}^{-1}C_2
\end{Equation}
这就解得
\begin{Equation}
    R_z=G_{m2}^{-1}(1+C_c^{-1}C_2)
\end{Equation}
整理如下
\begin{BoxFormula}[两级运放--调零电阻的设置]
    若要令$\omega_{z1}$为无穷大,调零电阻$R_z$应设置为
    \begin{Equation}
        R_z=G_{m2}^{-1}
    \end{Equation}
    若要令$\omega_{z1}=\omega_{p2}$,调零电阻$R_z$应设置为
    \begin{Equation}
        R_z=G_{m2}^{-1}(1+C_c^{-1}C_2)
    \end{Equation}
\end{BoxFormula}

接下来我们考虑$\omega_{z1}=\omega_{p2}$相互抵消的背景下,$\omega_{p1},\omega_{p3}$对相位裕度的影响
\begin{Equation}
    \phi_m=\pi-\arctan(\frac{\te{GB}}{|\omega_{p1}|})-\arctan(\frac{\te{GB}}{|\omega_{p3}|})
\end{Equation}
和之前一样,第一项近似为$\pi/2$
\begin{Equation}
    \phi_m=\pi/2-\arctan(\frac{\te{GB}}{|\omega_{p3}|})
\end{Equation}
我们同样用$k_{p3}$分别表示$|\omega_{p3}|$相对$\te{GB}$的倍数
\begin{Equation}
    |\omega_{p3}|=k_{p3}\te{GB}
\end{Equation}
这里$\omega_{p3}$有些类似于无调零电阻时的$\omega_{p2}$,但这一次,不再有零点的影响了。
\begin{BoxFormula}[两级运放--调零电阻--相位裕度和极点频率]
    若要求$\phi_m=\pi/4$,且调零电阻保证$\omega_{z1}=\omega_{p2}$,则应有
    \begin{Equation}
        |\omega_{p3}|=1.00\te{GB}\qquad k_{p3}=1.00
    \end{Equation}
    若要求$\phi_m=\pi/3$,且调零电阻保证$\omega_{z1}=\omega_{p2}$,则应有
    \begin{Equation}
        |\omega_{p2}|=1.73\te{GB}\qquad k_{p2}=1.73
    \end{Equation}
\end{BoxFormula}
在\xref{fig:在使用调零电阻的米勒补偿下相位裕度和极点的关系}中,绘制了$k_{p3}$关于$\phi_m$的曲线。
\begin{Figure}[在使用调零电阻的米勒补偿下相位裕度和极点的关系]
    \includegraphics[scale=0.8]{build/Chapter05B_08a.fig.pdf}
\end{Figure}

根据$|\omega_{p3}|=k_{p3}\te{GB}$,代入\xref{fml:两级运放--调零电阻--相位裕度和极点频率}和\xref{fml:两级运放--米勒补偿--单位增益带宽}
\begin{Equation}
    R_z^{-1}C_1^{-1}=k_{p3}G_{m1}C_c^{-1}
\end{Equation}
代入\xref{fml:两级运放--调零电阻的设置}中$\omega_{z1}=\omega_{p2}$下的$R_z$
\begin{Equation}
    G_{m2}(1+C_c^{-1}C_2)^{-1}C_1^{-1}=k_{p3}G_{m1}C_c^{-1}
\end{Equation}
假设$C_2\gg C_c$,忽略左侧括号内的$1$
\begin{Equation}
    G_{m2}C_cC_2^{-1}C_1^{-1}=k_{p3}G_{m1}C_c^{-1}
\end{Equation}
整理得到
\begin{Equation}
    C_c^2=k_{p3}G_{m1}G_{m2}^{-1}C_1C_2
\end{Equation}
即
\begin{Equation}
    C_c=\sqrt{k_{p3}G_{m1}G_{m2}^{-1}C_1C_2}
\end{Equation}
我们将结果整理如下
\begin{BoxFormula}[两级运放--调零电阻--跨接电容]
    若要保证$|\omega_{p3}|=k_{p3}\te{GB}$,则跨接电容$C_c$需要满足
    \begin{Equation}
        C_c=\sqrt{k_{p3}G_{m1}G_{m2}^{-1}C_1C_2}
    \end{Equation}
\end{BoxFormula}
