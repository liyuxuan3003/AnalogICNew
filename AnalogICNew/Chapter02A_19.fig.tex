\documentclass{xStandalone}

\begin{document}
\begin{tikzpicture}[scale=1.2]

\newcounter{cnt}

\xSetX
\setcounter{cnt}{0}
\foreach \x in {A,B,C,D,E,F}
{
    \xValDefine{x\x L}[\thecnt*1.8+0.0]
    \xValDefine{x\x C}[\thecnt*1.8+0.5]
    \xValDefine{x\x R}[\thecnt*1.8+1.0]
    \stepcounter{cnt}
}

\xSetY
\setcounter{cnt}{0}
\foreach \x in {A,B,C,D,E,F}
{
    \xValDefine{y\x A}[-\thecnt*1.5+0.0]
    \xValDefine{y\x C}[-\thecnt*1.5-0.5]
    \xValDefine{y\x B}[-\thecnt*1.5-1.0]
    \stepcounter{cnt}
}

\draw[fill=mnmP0] (xAL|-yAA) rectangle (xAR|-yAB);
\draw[fill=mnmP1] (xBL|-yAA) rectangle (xBR|-yAB);
\draw[fill=mnmP2] (xCL|-yAA) rectangle (xCR|-yAB);
\draw[fill=mnmP3] (xDL|-yAA) rectangle (xDR|-yAB);

\draw[fill=mnmN0] (xAL|-yBA) rectangle (xAR|-yBB);
\draw[fill=mnmN1] (xBL|-yBA) rectangle (xBR|-yBB);
\draw[fill=mnmN2] (xCL|-yBA) rectangle (xCR|-yBB);
\draw[fill=mnmN3] (xDL|-yBA) rectangle (xDR|-yBB);

\draw[fill=mnmSiO2] (xAL|-yCA) rectangle (xAR|-yCB);

\draw[fill=mnmSi3N4] (xDL|-yCA) rectangle (xDR|-yCB);

\draw[fill=mnmPolySi] (xAL|-yDA) rectangle (xAR|-yDB);

\draw[fill=mnmTiSi2] (xDL|-yDA) rectangle (xDR|-yDB);

\draw[fill=mnmAl] (xAL|-yEA) rectangle (xAR|-yEB);

\draw[fill=mnmW] (xDL|-yEA) rectangle (xDR|-yEB);

\draw[fill=mnmLig] (xAL|-yFA) rectangle (xAR|-yFB);

\path (xEL|-yAC) node[right] {P\hspace{0.4em}型硅~~\cx{Si}};
\path (xEL|-yBC) node[right] {N型硅~~\cx{Si}};

\path (xBL|-yCC) node[right] {氧化硅~~\cx{SiO2}};
\path (xEL|-yCC) node[right] {氮化硅~~\cx{Si3N4}};

\path (xBL|-yDC) node[right] {多晶硅~~\cx{Poly-Si}};
\path (xEL|-yDC) node[right] {硅化钛~~\cx{TiSi2}};

\path (xBL|-yEC) node[right] {铝~~\cx{Al}};
\path (xEL|-yEC) node[right] {钨~~\cx{W}};

\path (xBL|-yFC) node[right] {光刻胶};

\xValBorder{xAL}{xFR}{yFB}{yAA}*

\end{tikzpicture}
\end{document}