\section{MOS电流镜}

\subsection{简单电流镜}
电流镜是由\xref{sec:MOS二极管}中的MOS二极管和\xref{sec:MOS电流漏}中的MOS电流漏组合而成的简单应用,电流镜的用途是将参考电流$I_{REF}$复制到输出支路电流$i_{OUT}$上并期望$i_{OUT}$尽可能不随$v_{OUT}$变化,如\xref{fig:简单电流镜}所示。电流镜的工作原理可以直接用\xref{fml:并联MOS管的特性关系}解释:两个过驱电压相等且尺寸相同的MOS管具有相同的电流。不过为充分考虑二阶效应和失配的影响,这里重做一遍分析。

\begin{Figure}[简单电流镜]
    \includegraphics[scale=0.8]{build/Chapter03D_07.fig.pdf}
\end{Figure}

我们可以将电流比$i_O/I_{REF}$写作
\begin{Equation}[电流镜的基本关系]
    \frac{i_{OUT}}{I_{REF}}=\frac{K_2'(W_2/L_2)(v_{GS}-V_{T2})^2(1+\lambda v_{DS2})}{K_1'(W_1/L_1)(v_{GS}-V_{T1})^2(1+\lambda v_{DS1})}
\end{Equation}
假如$K_1'=K_2'$且$V_{T1}=V_{T2}$并忽略沟道调制,即有
\begin{Equation}
    \frac{i_{OUT}}{I_{REF}}=\frac{(W_2/L_2)}{(W_1/L_1)}
\end{Equation}
这就是电流镜的基本原理,$M_2$与$M_1$的尺寸比确定了$i_O/I_{REF}$的复制比
\begin{BoxFormula}[MOS电流镜的大信号特性]
    MOS电流镜的大信号特性为
    \begin{Equation}
        \frac{i_{OUT}}{I_{REF}}=\frac{(W_2/L_2)}{(W_1/L_1)}
    \end{Equation}
\end{BoxFormula}
在电流镜中,过驱动电压$V_{ON}$是通过MOS二极管由参考电流$I_{REF}$确定的,通过\xref{fig:MOS二极管的大信号特性}或者\xref{fig:MOS电流漏的大信号特性}都可以看出:$V_{ON}=\SI{1.0}{V}$的过驱电压大致对应$\SI{0.06}{mA}$的电流。由于先前\xref{sec:MOS电流漏}的仿真都是按$V_{ON}=\SI{1.0}{V}$进行,为了保持$V_{ON}$的一致性以及确保$V_{ON}$是$\SI{1.0}{V}$这样的一个简单整数,这里的仿真取参考电流为$I_{REF}=\SI{0.06}{mA}$(而不是$\SI{0.05}{mA}$或$\SI{0.10}{mA}$的整数)。

\xref{fig:简单电流镜的大信号特性}是对\xref{fig:简单电流镜}所示电路取$I_{REF}=\SI{0.6}{mA}$的仿真,其中,令$M_1,M_2$管的宽长比相等。

解读仿真结果前,我们先阐释下为什么用这三张图像分析电流镜特性
\begin{itemize}
    \item \xref{fig:简单电流镜--电流特性}展示了$I_{REF}$和$i_{OUT}$随$v_{OUT}$的变化趋势,这直观反映了电流特性。
    \item \xref{fig:简单电流镜--漏极电压}展示了各管的漏端电压$v_D$,这恰好覆盖了电流镜中所有节点。
    \item \xref{fig:简单电流镜--M2管工作区分析}展示了非二极管接法(因为二极管接法的MOS管始终处于饱和区,无分析的必要)的MOS管的$v_{DS}$和$v_{GS}-V_T$,分别用紫线和绿线表示。若紫线高于绿线,则表明处于饱和区,这有助于我们掌握工作区的变化。更复杂的电流镜中这种图像不止一张。
\end{itemize}
之后,我们将总是用这三种图像分析电流镜的大信号特性。
\begin{Figure}[简单电流镜的大信号特性]
    \begin{FigureSub}[电流特性;简单电流镜--电流特性]
        \includegraphics[scale=0.6]{build/Chapter03D_01_0.fig.pdf}
    \end{FigureSub}
    \begin{FigureSub}[漏极电压;简单电流镜--漏极电压]
        \includegraphics[scale=0.6]{build/Chapter03D_01_1.fig.pdf}
    \end{FigureSub}\\ \vspace{0.25cm}
    \begin{FigureSub}[$M_2$管工作区分析;简单电流镜--M2管工作区分析]
        \includegraphics[scale=0.6]{build/Chapter03D_01_2.fig.pdf}
    \end{FigureSub}
\end{Figure}

\xref{fig:简单电流镜的大信号特性}中,我们注意到
\begin{itemize}
    \item 简单电流镜输出端$v_{out}$的最小电压$V_{\min}=V_{ON}=\SI{1}{V}$,该值恰好能使$M_2$饱和。
    \item 简单电流镜和简单电流漏类似,其特性受到沟道调制效应的影响,并不是特别理想。我们知道,只有当$v_{DS2}=v_{DS1}$时才有$i_{OUT}=I_{REF}$。这里$v_{DS1}=V_{ON}+V_T=\SI{1.7}{V}$,而我们确实观察到$v_{DS2}=v_{OUT}=V_{ON}=\SI{1.0}{V}$取最小值时$i_{OUT}$是略小于$I_{REF}$的,只有当$v_{DS2}=v_{OUT}$增大到$\SI{1.7}{V}$时$i_{OUT}$才与$I_{REF}$线相交(电流相等不在$V_{\min}$时取)。
    \item \xref{fig:简单电流镜--漏极电压} $v_{D1}$恒定为$\SI{1.7}{V}$是$V_{ON}+V_T$,$v_{D2}$就是$v_{OUT}$。
\end{itemize}
\begin{BoxFormula}[简单电流镜的最小输出电压]
    简单电流镜的最小输出电压为
    \begin{Equation}
        V_{\min}=V_{ON}
    \end{Equation}
\end{BoxFormula}

我们或许会关心电流镜在小信号下的输出电阻$r_{out}$是多少?不过,由于电流镜的左右两侧是完全独立的,电流镜的$r_{out}$就是其右侧电流漏的$r_{out}$,可直接引用\xref{fml:MOS电流漏的输出电阻}的结果。

在改进简单电流镜前,我们还想就其讨论下各类非理想因素对电流镜工作的影响。

\subsection{简单电流镜中的非理想因素}
使电流镜偏离\xref{fml:MOS电流镜的大信号特性}的理想特性的影响因素,有以下三种
\begin{enumerate}
    \item 沟长调制,即考虑$\lambda$时$v_{DS1}\neq v_{DS2}$的影响。
    \item 参数失配,即考虑两管具有不相同的阈值电压$V_{T}$和跨导增益$K'$。
    \item 几何失配,即考虑两管具有偏离预设的宽长比$(W/L)$。
\end{enumerate}
我们主要对沟长调制和参数调制做定量分析,分析时,均假定$M_1,M_2$具有相同的宽长比。

\subsubsection{沟长调制}
分析沟长调制的影响,根据\xref{eq:电流镜的基本关系}
\begin{Equation}
    \frac{i_{OUT}}{I_{REF}}=\frac{1+\lambda v_{DS2}}{1+\lambda v_{DS1}}
\end{Equation}
整理如下
\begin{BoxFormula}[电流镜受沟长调制的影响]
    电流镜受沟长调制的影响可以表示为
    \begin{Equation}
        \frac{i_{OUT}}{I_{REF}}=\frac{1+\lambda v_{DS2}}{1+\lambda v_{DS1}}
    \end{Equation}
\end{BoxFormula}
\xref{fig:沟长调制对电流镜的影响}展示了电流比误差$(i_{OUT}/I_{REF})-1$和漏极电压差$v_{DS2}-v_{DS1}$的关系,请注意,这仍然与$I_{REF}$有关,因为$I_{REF}$会决定$v_{GS1}$而$v_{GS1}=v_{DS1}$。从图像中我们可以看出
\begin{itemize}
    \item $I_{REF}$的影响很小,这是因为$\lambda$很小的缘故。换言之,电流镜因沟道调制造成的电流比误差$(i_{OUT}/I_{REF})-1$几乎只由电压差$v_{DS2}-v_{DS1}$决定,而与$v_{DS1}$即$I_{REF}$关系不大。
    \item 电流比误差$(i_{OUT}/I_{REF})-1$随$v_{DS2}-v_{DS1}$的增大而增大。
    \item 电压差$v_{DS2}-v_{DS1}$的下限是$-V_T$即$\SI{-0.7}{V}$,低于这个值后$M_2$将无法饱和。
\end{itemize}
\begin{Figure}[沟长调制对电流镜的影响]
    \includegraphics[scale=0.8]{build/Chapter03D_05c.fig.pdf}
\end{Figure}

\subsubsection{参数失配}\setpeq{电流镜参数失配}
分析参数失配的影响,根据\xref{eq:电流镜的基本关系}
\begin{Equation}&[1]
    \frac{i_{OUT}}{I_{REF}}=\frac{K_2'(v_{GS}-V_{T2})^2}{K_1'(v_{GS}-v_{T1})^2}
\end{Equation}
诚然,这个公式已经正确的表达了$K_1'\neq K_2'$以及$V_{T1}\neq V_{T2}$的影响了,但是我们期望这样一个公式:已知$K'$和$V_T$因工艺可能的波动范围,估测其对电流比误差的总影响。这就要求我们进行一些代换和近似。第一步要做的,是应用差分的方式重新表示$K_1',K_2'$以及$V_{T1},V_{T2}$
\begin{Gather}
    K_1'=K'-0.5\delt{K'}\xlabelpeq{2a}\\
    K_2'=K'+0.5\delt{K'}\xlabelpeq{2b}\\
    V_{T1}=V_T-0.5\delt{V_T}\xlabelpeq{2c}\\
    V_{T2}=V_T+0.5\delt{V_T}\xlabelpeq{2d}
\end{Gather}
其中,$K',\delt{K'}$是$K_2,K_1$的均值和差,$V_T,\delt{V_T}$是$V_{T2},V_{T1}$的均值和差。作为估计,我们可以认为这里的$K',V_T$取相应默认参数,而$\delt{K}',\delt{V_T}$是波动范围,基于此考察电流比误差。

将\xrefpeq{2a},\xrefpeq{2b},\xrefpeq{2c},\xrefpeq{2d}代入\xrefpeq{1},得到
\begin{Equation}
    \frac{i_{OUT}}{I_{REF}}=\frac{(K'+0.5\delt K')(v_{GS}-V_T-0.5\delt{V_T})}{(K'-0.5\delt K')(v_{GS}-V_T+0.5\delt{V_T})}
\end{Equation}
提出$K'$和$v_{GS}-V_T$并约掉
\begin{Equation}
    \frac{i_{OUT}}{I_{REF}}=\frac{[1+0.5\delt{K'}/K'][1-0.5\delt V_T/(v_{GS}-V_T)]^2}{[1-0.5\delt{K'}/K'][1+0.5\delt V_T/(v_{GS}-V_T)]^2}
\end{Equation}
由于$1$后面的变量都很小,适用$1/(1-x)\approx 1+x$和$1/(1+x)\approx 1-x$的近似
\begin{Equation}
    \frac{i_{OUT}}{I_{REF}}=\qty[1+\frac{\delt{K'}}{2K'}]^2\qty[1-\frac{\delt{V_T}}{2(v_{GS}-V_T)}]^4
\end{Equation}
仅保留一阶项,我们知道$(1+x)^2=x^2+2x+1$和$(1+x)^4=x^4+4x^3+4x^2+1$
\begin{Equation}
    \frac{i_{OUT}}{I_{REF}}=1+\frac{\delt{K'}}{K'}-\frac{2\delt{V_T}}{v_{GS}-V_T}
\end{Equation}
\begin{BoxFormula}[电流镜受参数失配的影响]
    电流镜受参数失配的影响可以表示为
    \begin{Equation}
        \frac{i_{OUT}}{I_{REF}}=1+\frac{\delt{K'}}{K'}-\frac{2\delt{V_T}}{v_{GS}-V_T}
    \end{Equation}
\end{BoxFormula}

至此,如果已知$\delt{K'}$和$\delt{V_T}$的值,估算误差就变得很容易了。

\begin{Figure}[参数失配对电流镜的影响]
    \begin{FigureSub}[跨导增益失配]
        \includegraphics[scale=0.8]{build/Chapter03D_05a.fig.pdf}
    \end{FigureSub}
    \begin{FigureSub}[阈值电压失配]
        \includegraphics[scale=0.8]{build/Chapter03D_05b.fig.pdf}
    \end{FigureSub}
\end{Figure}

\xref{fig:参数失配对电流镜的影响}展示了参数失配的影响
\begin{itemize}
    \item 当$\delt{K}'$为正时,电流比误差为正,电流会偏大。
    \item 当$\delt{V_T}$为正时,电流比误差为负,电流会偏小。
    \item 当$\delt{V_T}$一定时,越大的参考电流$I_{REF}$意味着越大的误差。
\end{itemize}

\subsubsection{几何失配}
关于几何失配,即宽长比的失配,更具体的分析需要考虑版图,不展开讨论。只需了解一个事实:对于$W,L$大于$\SI{10}{um}$的晶体管,几何失配的误差相较沟长调制和参数失配可以忽略。

\subsection{共源共栅电流镜}
共源共栅电流镜是对简单电流镜的改进,其输出侧 改用了共源共栅电流漏,电路如\xref{fig:共源共栅电流镜}所示。\goodbreak

\begin{Figure}[共源共栅电流镜]
    \includegraphics{build/Chapter03D_08.fig.pdf}
\end{Figure}

我们先来推定所有节点的电压$v_D$,这会最终给出输出电压$v_{OUT}$的最小值$V_{\min}$
\begin{enumerate}
    \item $M_1,M_2,M_3,M_4$栅源电压均为$v_{GS}=V_{ON}+V_T$。
    \item $v_{D1}=V_{ON}+V_T$:$M_1$二极管连接,故$v_{DS1}=v_{GS1}=V_{ON}+V_T$,$v_{D1}=v_{DS1}$。
    \item $v_{D3}=2V_{ON}+2V_T$:$M_3$二极管连接,故$v_{DS3}=v_{GS3}=V_{ON}+V_T$,$v_{D3}=v_{DS3}-v_{D1}$。
    \item $v_{D2}=V_{ON}+V_T$:$M_4$满足$v_{GS4}=V_{ON}+V_T$,$v_{D2}=v_{D3}-v_{GS4}$。
    \item $v_{D4}=2V_{ON}+V_T$:$M_4$饱和所需的最小$v_{DS4}=V_{ON}$,$v_{D4}=v_{D2}+v_{DS4}$。
\end{enumerate}
由此可见,应有$V_{\min}=2V_{ON}+V_T$,然而,这个$V_{\min}$值太高了。我们知道,共源共栅电流漏只需要$V_{\min}=2V_{ON}$,共源共栅电流镜明显浪费了一个阈值电压$V_T$,这是为什么呢?我们还记得,共源共栅电流漏中,$V_{G1}=V_{ON}+V_T$,$V_{G2}\geq 2V_{ON}+V_T$,这里的$V_{G1},V_{G2}$是由左侧二极管连接的MOS管提供,但我们注意到,这里$V_{G2}=v_{D3}=2V_{ON}+2V_T$,这就将$v_{D2}$钳至$v_{D2}=V_{ON}+V_T$,而$v_{D2}$所需的令$M_2$饱和的最低值是$v_{D2}=V_{ON}$,相应的,其上的$v_{D4}$需要的最低值也由$2V_{ON}$上升至$2V_{ON}+V_T$,这就是为何$V_{\min}$浪费了一个阈值$V_T$的原因。
\begin{BoxFormula}[共源共栅电流镜的最小输出电压]
    共源共栅电流镜的最小输出电压为
    \begin{Equation}
        V_{\min}=2V_{ON}+V_T
    \end{Equation}
\end{BoxFormula}
因此,下一步的改进,就是要想办法将$v_{D3}=2V_{ON}+2V_T$减小至$v_{D3}=2V_{ON}+V_T$。

\begin{Figure}[共源共栅电流镜的大信号特性]
    \begin{FigureSub}[电流特性;共源共栅电流镜--电流特性]
        \includegraphics[scale=0.6]{build/Chapter03D_02_0.fig.pdf}
    \end{FigureSub}
    \begin{FigureSub}[漏极电压;共源共栅电流镜--漏极电压]
        \includegraphics[scale=0.6]{build/Chapter03D_02_1.fig.pdf}
    \end{FigureSub}\\ \vspace{0.25cm}
    \begin{FigureSub}[$M_2$管工作区分析;共源共栅电流镜--M2管工作区分析]
        \includegraphics[scale=0.6]{build/Chapter03D_02_2.fig.pdf}
    \end{FigureSub}
    \begin{FigureSub}[$M_4$管工作区分析;共源共栅电流镜--M4管工作区分析]
        \includegraphics[scale=0.6]{build/Chapter03D_02_3.fig.pdf}
    \end{FigureSub}
\end{Figure}

\xref{fig:共源共栅电流镜的大信号特性}展示了共源共栅电流镜的大信号特性
\begin{itemize}
    \item 共源共栅电流镜的最小电压$V_{\min}=2V_{ON}+V_T=\SI{2.7}{V}$,与理论相符。
    \item 共源共栅电流镜在$v_{OUT}=V_{\min}$时$M_4$饱和,$M_2$饱和先于其发生。
    \item 共源共栅电流镜的电流特性改善了很多,这是得益于共源共栅电流漏的高输出阻抗。
\end{itemize}

\subsection{高摆幅共源共栅电流镜--基础型}
共源共栅电流镜的问题主要是$V_{\min}$过高,且理论上有一个阈值$V_T$的改进空间,这一小节提出的高摆幅共源共栅电流镜就试图解决这个问题,其电路如\xref{fig:高摆幅共源共栅电流镜--基础型}所示,相较\xref{fig:共源共栅电流镜},其变化在于将$M_1,M_3$管分别置于两个独立的$I_{REF}$支路上,且将$M_3$的宽长比改为$1/4$
\begin{itemize}
    \item 按照\xref{fml:串联MOS管的特性关系},尺寸之比(宽长比之比)等于过驱电压平方之反比。
    \item $M_1$宽长比为$1/1$,设其有$V_{ON}$过驱电压,故$v_{GS1}=V_{ON}+V_T$。
    \item $M_3$宽长比为$1/4$,则应有$2V_{ON}$过驱电压,故$v_{GS3}=2V_{ON}+V_T$。
\end{itemize}
现在,弄清$v_{GS3}$的变化后,我们推定所有节点的电压$v_D$
\begin{enumerate}
    \item $M_1,M_2,M_4$的$v_{GS}=V_{ON}+V_T$,$M_3$的$v_{GS}=2V_{ON}+V_T$。
    \item $v_{D1}=V_{ON}+V_T$:$M_1$二极管连接,故$v_{DS1}=v_{GS1}=V_{ON}+V_T$,$v_{D1}=v_{DS1}$。
    \item $v_{D3}=2V_{ON}+V_T$:$M_3$二极管连接,故$v_{DS3}=v_{GS3}=2V_{ON}+V_T$,$v_{D3}=v_{DS3}$。
    \item $v_{D2}=V_{ON}$:$M_4$满足$v_{GS4}=V_{ON}+V_T$,$v_{D2}=v_{D3}-v_{GS4}$。
    \item $v_{D4}=2V_{ON}$:$M_4$饱和所需的最小$v_{DS4}=V_{ON}$,$v_{D4}=v_{D2}+v_{DS4}$。
\end{enumerate}
\begin{Figure}[高摆幅共源共栅电流镜--基础型]
    \includegraphics[scale=0.8]{build/Chapter03D_09.fig.pdf}
\end{Figure}
由此可见,高摆幅共源共栅电流镜实现高摆幅的思路,就是将$M_3$和$M_1$分离,通过调节$M_3$的尺寸使其具有能使$M_2$饱和所需的最小的$v_{D3}=2V_{ON}+V_T$,从而降低至$V_{\min}=2V_{ON}$。

\begin{BoxFormula}[高摆幅共源共栅电流镜的最小输出电压]
    高摆幅共源共栅电流镜的最小输出电压为
    \begin{Equation}
        V_{\min}=2V_{ON}
    \end{Equation}
\end{BoxFormula}

\xref{fig:共源共栅电流镜的大信号特性}展示了高摆幅共源共栅电流镜--基础型的大信号特性
\begin{itemize}
    \item 最小输出电压$V_{\min}=2V_{ON}+V_T=\SI{2.0}{V}$,实现了高摆幅。
    \item 最小输出电压$V_{\min}=2V_{ON}+V_T=\SI{2.0}{V}$的理论值似乎并不准确,理论上$v_{out}=V_{\min}$时$M_2,M_4$管应同时饱和,而实际上,$M_4$管饱和发生的稍早了些,$M_2$甚至根本没有饱和!该问题和\xref{subsec:共源共栅电流漏}分析共源共栅电流漏时遇到的问题是相同的,其本质是,高摆幅共源共栅卡了$M_2$的饱和边界,而体效应使$M_2$的$v_{D2}$比预期低了些,就落入了线性区。解决方案很简单,只要略再减小些$M_3$的宽长比(比如$1/4$至$1/5$)使$v_{D3}$稍大些即可。当然,这会稍微损失些摆幅,不过相较普通的共源共栅,这仍然是高摆幅的。
    \item 电流特性出现了一些问题!注意到$i_{OUT}$变得略小于$I_{REF}$,这是因为$v_{DS1}\neq v_{DS2}$导致的沟道调制误差,\xref{fig:高摆幅共源共栅电流镜--基础型}中我们可以看到,$v_{DS1}=v_{D1}=V_{ON}+V_T$,$v_{DS2}=v_{D2}=V_{ON}$。
\end{itemize}

\begin{Figure}[高摆幅共源共栅电流镜--基础型的大信号特性]
    \begin{FigureSub}[电流特性;高摆幅共源共栅电流镜--基础型--电流特性]
        \includegraphics[scale=0.6]{build/Chapter03D_03_0.fig.pdf}
    \end{FigureSub}
    \begin{FigureSub}[漏极电压;高摆幅共源共栅电流镜--基础型--漏极电压]
        \includegraphics[scale=0.6]{build/Chapter03D_03_1.fig.pdf}
    \end{FigureSub}\\ \vspace{0.25cm}
    \begin{FigureSub}[$M_2$管工作区分析;高摆幅共源共栅电流镜--基础型--M2管工作区分析]
        \includegraphics[scale=0.6]{build/Chapter03D_03_2.fig.pdf}
    \end{FigureSub}
    \begin{FigureSub}[$M_4$管工作区分析;高摆幅共源共栅电流镜--基础型--M4管工作区分析]
        \includegraphics[scale=0.6]{build/Chapter03D_03_3.fig.pdf}
    \end{FigureSub}
\end{Figure}



因此下一步的改进,就是要设法令$v_{DS1}=v_{DS2}$,保证电流镜复制的准确性。

\subsection{高摆幅共源共栅电流镜--改进型}
改进的高摆幅共源共栅电流镜如\xref{fig:高摆幅共源共栅电流镜--改进型}所示,相较原先的\xref{fig:高摆幅共源共栅电流镜--基础型}的电路,改进型又引入了一个晶体管$M_5$,置于$M_1$上方,且原先$M_1$是二极管连接的,其栅与漏短接的,而在现在的改进型中,其栅改与$M_5$的漏短接。同时,$M_5$的栅和$M_3,M_4$的栅相连(后两者原本就相连)。

\begin{Figure}[高摆幅共源共栅电流镜--改进型]
    \includegraphics[scale=0.8]{build/Chapter03D_10.fig.pdf}
\end{Figure}

现在,我们推定所有节点的电压$v_D$
\begin{enumerate}
    \item $M_1,M_2,M_4,M_5$的$v_{GS}=V_{ON}+V_T$,$M_3$的$v_{GS}=2V_{ON}+V_T$。
    \item $v_{D3}=2V_{ON}+V_T$:$M_3$二极管连接,故$v_{DS3}=v_{GS3}=2V_{ON}+V_T$,$v_{D3}=v_{DS3}$。
    \item $v_{D5}=V_{ON}+V_T$:$M_1$满足$v_{GS1}=V_{ON}+V_T$,$v_{D5}=v_{GS1}$。
    \item $v_{D1}=V_{ON}$:$M_5$满足$v_{GS5}=V_{ON}+V_T$,$v_{D1}=v_{D3}-v_{GS5}$。
    \item $v_{D2}=V_{ON}$:$M_4$满足$v_{GS4}=V_{ON}+V_T$,$v_{D2}=v_{D3}-v_{GS4}$。
    \item $v_{D4}=2V_{ON}$:$M_4$饱和所需的最小$v_{DS4}=V_{ON}$,$v_{D4}=v_{D2}+v_{DS4}$。
\end{enumerate}
由此可见,$M_5$引入的目的,就是通过$v_{GS5}=V_{ON}+V_T$强令位于其源端的$v_{D1}=v_{D3}-v_{GS5}$变为$v_{D1}=V_{ON}$,这样就和$v_{D2}=V_{ON}$一致,实现了$v_{DS1}=v_{DS2}$,避免了因为沟道调制导致的$i_{OUT}\neq I_{REF}$的误差。\xref{fig:高摆幅共源共栅电流镜--改进型--电流特性}亦验证了改进后$i_{OUT}\neq I_{REF}$的问题确实得到了改善。

\begin{Figure}[高摆幅共源共栅电流镜--改进型的大信号特性]
    \begin{FigureSub}[电流特性;高摆幅共源共栅电流镜--改进型--电流特性]
        \includegraphics[scale=0.6]{build/Chapter03D_04_0.fig.pdf}
    \end{FigureSub}
    \begin{FigureSub}[漏极电压;高摆幅共源共栅电流镜--改进型--漏极电压]
        \includegraphics[scale=0.6]{build/Chapter03D_04_1.fig.pdf}
    \end{FigureSub}\\ \vspace{0.25cm}
    \begin{FigureSub}[$M_2$管工作区分析;高摆幅共源共栅电流镜--改进型--M2管工作区分析]
        \includegraphics[scale=0.6]{build/Chapter03D_04_2.fig.pdf}
    \end{FigureSub}
    \begin{FigureSub}[$M_4$管工作区分析;高摆幅共源共栅电流镜--改进型--M4管工作区分析]
        \includegraphics[scale=0.6]{build/Chapter03D_04_3.fig.pdf}
    \end{FigureSub}
\end{Figure}
在本节结尾,我们做一个总结,以梳理电流镜结构演进中到改善了什么性能。
\begin{Tablex}[电流镜的特性总结]{lllX}
<电流镜类型&$V_{min}$&是否输出电阻较大?&是否不受沟道调制影响?\\>
简单&$V_{ON}$&否&否($v_{D1}=V_{ON}+V_T, v_{D3}=v_{OUT}$)\\
共源共栅&$2V_{ON}+V_T$&是&是($v_{D1}=V_{ON}+V_T, v_{D3}=V_{ON}+V_T$)\\
高摆幅共源共栅--基础型&$2V_{ON}$&是&否($v_{D1}=V_{ON}+V_T, v_{D3}=V_{ON}$)\\
高摆幅共源共栅--改进型&$2V_{ON}$&是&是($v_{D1}=V_{ON}\hspace{2.5em}, v_{D3}=V_{ON}$)\\
\end{Tablex}