\section{MOS电流漏}

\subsection{简单电流漏}
本节将介绍MOS电流漏和MOS电流源,这种名称的区别取决于使用NMOS还是PMOS
\begin{itemize}
    \item 电流漏对应NMOS器件,其位于最下方,是电流“漏出”的地方。
    \item 电流源对应PMOS器件,其位于最上方,是电流“源起”的地方。
\end{itemize}
\begin{Figure}[MOS电流漏或电流源]
    \begin{FigureSub}[NMOS电流漏]
        \qquad
        \includegraphics[scale=0.8]{build/Chapter03C_03.fig.pdf}
        \qquad
    \end{FigureSub}
    \begin{FigureSub}[PMOS电流源]
        \qquad
        \includegraphics[scale=0.8]{build/Chapter03C_04.fig.pdf}
        \qquad
    \end{FigureSub}
\end{Figure}
以下我们均以NMOS电流漏为例进行研究。

简单来说,电流漏能在一定栅压$v_G=V_G$偏置下,在漏源两端间提供一个相对稳定的电流。当然,前提是MOS管必须处于饱和区,因此输出端电压最小为$v_{OUT}\geq V_G-V_T$。电流漏是最直观反映“饱和区MOS管相当于电流源\footnote{这里的“电流源”是通常意义上的电流源,与“电压源”相对,不应当与“PMOS电流源”混淆。}”的CMOS子电路,我们可以很容易的写出其方程
\begin{BoxFormula}[MOS电流漏的大信号特性]
    MOS电流漏的大信号特性为
    \begin{Equation}
        i_{OUT}=\frac{\beta}{2}(V_{G}-V_T)^2(1+\lambda v_{OUT})\qquad v_{OUT}\geq V_G-V_T
    \end{Equation}
\end{BoxFormula}

\xref{fig:MOS电流漏的大信号特性}展示了MOS电流漏的大信号特性,取$V_{G}=\SI{1.7}{V}$,对应过驱电压$V_{ON}=\SI{1.0}{V}$
\begin{Figure}[MOS电流漏的大信号特性]
    \includegraphics[scale=0.6]{build/Chapter03C_01_0.fig.pdf}
\end{Figure}
若我们对比一下MOS二极管的\xref{fig:MOS二极管的大信号特性}和\xref{fig:MOS电流漏的大信号特性},某种意义上,我们可以称
\begin{itemize}
    \item MOS二极管的特性曲线,代表了MOS的转移特性曲线。
    \item MOS电流漏的特性曲线,代表了MOS的输出特性曲线。
\end{itemize}

\xref{fig:MOS电流漏的小信号电路}是MOS电流漏的小信号电路,由于栅接偏置且漏接地,其仅包含$g_{ds}$
\begin{Figure}[MOS电流漏的小信号电路]
    \includegraphics[scale=0.8]{build/Chapter03C_07.fig.pdf}
\end{Figure}

列出方程\setpeq{MOS电流漏小信号}
\begin{Equation}&[1]
    i_{out}=g_{ds}v_{ds}
\end{Equation}
这里端电压可以表示为
\begin{Equation}&[2]
    \begin{pmatrix}
        v_{gs}\\
        v_{bs}\\
        v_{ds}
    \end{pmatrix}=
    \begin{pmatrix}
        0\\
        0\\
        v_{out}
    \end{pmatrix}
\end{Equation}
将\xrefpeq{2}代入\xrefpeq{1}
\begin{Equation}
    i_{out}=g_{ds}v_{out}
\end{Equation}
因此
\begin{Equation}
    r_{out}=\frac{v_{out}}{i_{out}}=g_{ds}^{-1}
\end{Equation}
这一结论是符合常识的,若不考虑沟道调制$g_{ds}=0$则$r_{out}=\infty$,构成理想电流源。

\begin{BoxFormula}[MOS电流漏的输出电阻]
    MOS电流漏的输出电阻为
    \begin{Equation}
        r_{out}=g_{ds}^{-1}
    \end{Equation}
\end{BoxFormula}



\subsection{共源共栅电流漏}
\xref{fig:NMOS电流漏}所示的简单电流漏存在诸多问题,从\xref{fig:MOS电流漏的大信号特性}中我们可以看出即便在$v_{OUT}\geq V_G-V_T$的饱和区,由于沟道调制,电流$i_{OUT}$仍明显随电压$v_{OUT}$增大而增大,或者,从\xref{fml:MOS电流漏的输出电阻}中亦可以看出简单电流漏的输出电阻$r_{out}$仅为$g_{ds}^{-1}$级数,这是远远不够的。由此可见,眼下电流漏的改进方向就是,如何增大输出电阻$r_{OUT}$的值?本小节的共源共栅电流漏将给出答案。
\begin{Figure}[MOS共源共栅电流漏]
    \includegraphics[scale=0.8]{build/Chapter03C_05.fig.pdf}
\end{Figure}

\xref{fig:MOS共源共栅电流漏}展示了共源共栅电流漏的结构,不必过分从组态的角度纠结这个结构为什么被称为“共源共栅”,只需要知道“共源共栅”往往与这样堆叠的MOS管联系起来即可。现在让我们考虑一下大信号的问题,这里我们不再列写大信号特性的表达式,这很复杂且毫无必要,而是将简单的思考一下这个电路是怎么工作的。\xref{subsec:简单电流漏}的经验告诉我们,若忽略沟道调制,在饱和区,$i_{OUT}$取决于过驱动电压$V_{ON}=V_G-V_{T}$,$i_{OUT}=\beta V_{ON}^2/2$。而在这里的共源共栅电流漏中,$M_1$和$M_2$串联,两者上的电流$i_{OUT}$相同,因此,两者的过驱电压必然是相等的
\begin{Equation}
    v_{GS1}=V_{ON}+V_{T}\qquad 
    v_{GS2}=V_{ON}+V_{T}
\end{Equation}

由于$v_{GS1}=V_{G1}$,故$V_{G1}=V_{ON}+V_T$被确定。假如和之前一样,取$V_{ON}=\SI{1.0}{V}$并考虑到阈值电压$V_{T0}=\SI{0.7}{V}$,那么,我们就可以得到$V_{G1}=\SI{1.7}{V}$,和之前的$V_G=\SI{1.7}{V}$完全一样。\goodbreak

现在的问题是:$M_2$的偏置$V_{G2}$应当取什么值?$M_1$如何保证其自身处于饱和区?这两个问题的共通基础问题是,显然输出节点$v_{D2}=v_{OUT}$,但中间节点$v_{D1}$的值该如何确定?唔,这是很难的问题。然而为了令$M_1$处于饱和区,我们可以确定至少应有$v_{D1}\geq V_{ON}$,而我们又知道$v_{GS2}=V_{G2}-v_{D1}$和$v_{GS2}=V_{ON}+V_T$,即得$V_{G2}\geq 2V_{ON}+V_T$。那如果我们令$M_1$恰好处于饱和区边缘,即有$V_{G2}=2V_{ON}+V_T$成立,这就得到$V_{G2}$可取的最小值是$V_{G2}=\SI{2.7}{V}$。

现已知道$v_{D1}\geq V_{ON}$能令$M_1$饱和,那么$v_{D2}=v_{OUT}\geq 2V_{ON}$才能确保$M_2$也饱和。

至此,我们稍稍总结一下已有的经验
\begin{itemize}
    \item 简单电流漏输出电压需满足$v_{OUT}\geq V_{ON}$。
    \item 共源共栅电流漏及输出电压需满足$v_{OUT}\geq 2V_{ON}$,注意到$v_{OUT}$所需的最小值抬升了。
    \item 共源共栅电流漏中$M_1$的偏置$V_{G1}=V_{ON}+V_T$,$V_{G1}$直接确定$V_{ON}$。
    \item 共源共栅电流漏中$M_2$的偏置$V_{G2}\geq 2V_{ON}+V_T$,$V_{G2}$的最小值保证$M_1$饱和。
\end{itemize}

那么,共源共栅需要付出这么多的代价:更多的管子,更高的最低输出电压,更复杂的电路分析,其到底换来了什么补偿优势呢?如\xref{fig:MOS电流漏的大信号特性--电流特性}所示,相较\xref{fig:MOS电流漏的大信号特性},可以看出共源共栅电流漏在饱和后,电流$i_{OUT}$几乎不会随电压$v_{OUT}$变化了!这实际是因为,共源共栅电流漏具有比简单电流漏大的多的输出电阻$r_{out}$,这一点我们稍后会证明。另外,如\xref{fig:MOS电流漏的大信号特性--漏极电压情况}所示,我们注意到中间节点$v_{D1}$是一个略小于$\SI{1.0}{V}$的值,但按理论分析$v_{D1}$应精确等于$\SI{1.0}{V}$,尽管仿真与理论分析相差不多,但这一误差是怎么来的呢?这是因为,$M_2$管的源未接地,$M_2$存在一定的体效应,其$V_T$实际比$V_{T0}=\SI{0.7}{V}$大,故$v_{D1}=V_{G2}-V_{ON}-V_T$必然是比我们所预期的$\SI{1.0}{V}$要小的。这一问题其实是相当重要的,由于$v_{D1}=\SI{1.0}{V}$是令$M_1$管饱和的下限,实际上按当前设计,$M_1$管工作在线性区。因此,若考虑体效应的影响,$M_2$管的偏置$V_{G2}$需要比$2V_{ON}+V_{T0}=\SI{2.7}{V}$更大一些,才能令$M_1$管饱和。这一问题稍后我们还会再一次遇到。
\begin{Figure}[MOS电流漏的大信号特性]
    \begin{FigureSub}[电流特性;MOS电流漏的大信号特性--电流特性]
        \includegraphics[scale=0.6]{build/Chapter03C_02_0.fig.pdf}
    \end{FigureSub}
    \begin{FigureSub}[漏极电压情况;MOS电流漏的大信号特性--漏极电压情况]
        \includegraphics[scale=0.6]{build/Chapter03C_02_1.fig.pdf}
    \end{FigureSub}
\end{Figure}\goodbreak

\xref{fig:MOS共源共栅电流漏的小信号电路}是MOS共源共栅电流漏的小信号电路,其中$v_{1}$是$M_1$管漏端电压
\begin{Figure}[MOS共源共栅电流漏的小信号电路]
    \includegraphics[scale=0.8]{build/Chapter03C_06.fig.pdf}
\end{Figure}
列出方程\setpeq{MOS共源共栅电流漏的小信号电路}
\begin{Gather}
    i_{out}=g_{ds1}v_{ds1}\xlabelpeq{1a} \\
    i_{out}=g_{m2}v_{gs2}+g_{bs2}v_{bs2}+g_{ds2}v_{ds2}
    \xlabelpeq{1b}
\end{Gather}
这里端电压可以表示为
\begin{Equation}&[2]
    \begin{pmatrix}
        v_{gs1}&v_{gs2}\\
        v_{bs1}&v_{bs2}\\
        v_{ds1}&v_{ds2}
    \end{pmatrix}=
    \begin{pmatrix}
        0&-v_1\\
        0&-v_1\\
        v_{1}&v_{out}-v_1
    \end{pmatrix}
\end{Equation}
将\xrefpeq{2}代入\xrefpeq{1a}和\xrefpeq{1b}
\begin{Gather}
    i_{out}=g_{ds1}v_{1}\xlabelpeq{3a} \\
    i_{out}=g_{m2}(-v_{1})+g_{bs2}(-v_{1})+g_{ds2}(v_{out}-v_1)
    \xlabelpeq{3b}
\end{Gather}
就\xrefpeq{3b}整理得到
\begin{Equation}&[4]
    i_{out}=g_{ds2}v_{out}-(g_{m2}+g_{bs2}+g_{ds2})v_1
\end{Equation}
在\xrefpeq{4}中代入由\xrefpeq{3a}得到的$v_1=i_{out}g_{ds1}^{-1}$
\begin{Equation}
    i_{out}\qty[1+g_{ds1}^{-1}(g_{m2}+g_{bs2}+g_{ds2})]=g_{ds2}v_{out}
\end{Equation}
因此
\begin{Equation}
    r_{out}=\frac{v_{out}}{i_{out}}=g_{ds2}^{-1}\qty[1+g_{ds1}^{-1}(g_{m2}+g_{bs2}+g_{ds2})]
\end{Equation}
这一结果也可以由$r_{out}=g_{ds2}^{-1}\qty[1+g_{ds1}^{-1}(g_{m2}+g_{bs2}+g_{ds2})]$近似为$r_{out}=g_{ds1}^{-1}g_{ds2}^{-1}g_{m2}$,这是通过$g_{m2}\gg g_{bs2},g_{ds2}$以及$1$很小的假设完成的。若对比一下简单电流漏的$r_{out}=g_{ds}^{-1}$的输出电阻,我们可以看出,共源共栅电流漏中$M_2$将$M_1$的输出电阻$g_{ds1}^{-1}$放大了$g_{ds2}^{-1}g_{m2}$倍,这一倍数是很可观的,也因此,共源共栅电流漏具有比简单电流漏好的多的大信号特性。由此可以归纳出的一个普遍原理是:MOS管会放大其源端的电阻至原先的$g_{ds}^{-1}g_m$倍。事实上,这里即便将$M_2$源端从$M_1$(相当于$r=g_{ds1}^{-1}$)换成一个普通的电阻$r$,该放大原理同样是生效的。
\begin{BoxFormula}[MOS共源共栅电流漏的输出电阻]
    MOS共源共栅电流漏的输出电阻为
    \begin{Equation}
        r_{out}=g_{ds2}^{-1}\qty[1+g_{ds1}^{-1}(g_{m2}+g_{bs2}+g_{ds2})]
    \end{Equation}
    近似式为
    \begin{Equation}
        r_{out}=g_{ds1}^{-1}g_{ds2}^{-1}g_{m2}
    \end{Equation}
\end{BoxFormula}

\subsection{MOS器件的偏置原理}
在\xref{subsec:共源共栅电流漏}中,曾讨论过应如何确定$V_{G1},V_{G2}$的偏置,本小节会将这种偏置原理一般化。

我们知道$v_{GS}$可以拆分为阈值电压$V_T$和过驱电压$V_{ON}$两部分
\begin{Equation}
    v_{GS}=V_{ON}+V_T
\end{Equation}
关于过驱电压$V_{ON}$,我们要认识到两点
\begin{itemize}
    \item 过驱电压$V_{ON}$是$v_{GS}$超过阈值电压的那一部分,有$V_{ON}=v_{GS}-V_T$成立。
    \item 过驱电压$V_{ON}$是$v_{DS}$能令其工作在饱和区的最小电压。
\end{itemize}
现在,试想有两个工作在饱和区的MOS管$M_1,M_2$,它们的电流分别可以表示为
\begin{Equation}
    i_{D1}=\frac{1}{2}\beta_1V_{ON1}^2\qquad
    i_{D2}=\frac{1}{2}\beta_2V_{ON2}^2
\end{Equation}
以$\beta=K'(W/L)$展开,假设两管类型一致具有相同的$K'$
\begin{Equation}
    i_{D1}=\frac{1}{2}K'(W_1/L_1)V_{ON1}^2\qquad
    i_{D2}=\frac{1}{2}K'(W_2/L_2)V_{ON2}^2
\end{Equation}
若$M_1,M_2$管“串联”,即具有相同的沟道电流,满足$i_{D1}=i_{D2}$,则有
\begin{BoxFormula}[串联MOS管的特性关系]
    两个串联的MOS管,尺寸之比等于过驱电压平方之反比
    \begin{Equation}
        \frac{(W_1/L_1)}{(W_2/L_2)}=\frac{V_{ON2}^2}{V_{ON1}^2}
    \end{Equation}
\end{BoxFormula}
若$M_1,M_2$管“并联”,即具有相同的栅源电压,假定$V_T$相同则等效于$V_{ON1}=V_{ON2}$,则有
\begin{BoxFormula}[并联MOS管的特性关系]
    两个并联的MOS管,尺寸之比等于沟道电流之比
    \begin{Equation}
        \frac{(W_1/L_1)}{(W_2/L_2)}=\frac{i_{D1}}{i_{D2}}
    \end{Equation}
\end{BoxFormula}
请注意!上述“串联”和“并联”并不一定要求两个管子是物理的连接在一起,即便不是连接在一起,电压和电流也可以设置为相等。\xref{fml:串联MOS管的特性关系}和\xref{fml:并联MOS管的特性关系}是一组非常重要的设计原理。