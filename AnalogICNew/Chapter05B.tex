\section{两级运算放大器}

% \subsection{两级运放的电路结构}
两级运算放大器是最简单的运算放大器,它的电路结构如\xref{fig:两级运算放大器}所示
\begin{itemize}
    \item 第一级是一个电流镜负载的差分放大器,参见\xref{subsec:电流镜负载差分放大器的大信号特性}。
    \item 第二级是一个电流源负载的反相放大器,参见\xref{subsec:电流源负载反相放大器的大信号特性}。
    \item 差分放大器包括$M_1,M_2,M_3,M_4,M_5$,值得注意的是,之前是用$M_1$作正输入端而$M_2$作负输入端,这时放大器具有正的差模增益。此处恰好反过来了,故差分放大器的增益和后一级的反相放大器的增益就都是负的了,这样一来,整个运放的增益就是正的。
    \item 反相放大器包括$M_6,M_7$,有些不一样的是,这里是用PMOS的$M_6$作为放大管,而作为NMOS的$M_7$却是负载管。这样设计可能是为了让$M_7$和$M_5$能共用同一偏置。
    \item 采用正电源$V_{DD}$和负电源$V_{SS}$供电(而不是$V_{DD}$和GND)。
\end{itemize}

应指出的是,这里“两级运放”中的“两级”不应当视为一种对级数的泛指(事实上,大部分无缓冲的运算放大器都是两级的)。“两级运放”的名称特指这种最基本的无缓冲运放结构。

\begin{Figure}[两级运算放大器]
    \includegraphics[scale=0.8]{build/Chapter05B_02.fig.pdf}    
\end{Figure}

% 没错!运算放大器并没有那么复杂,不是吗?完全是由我们最熟悉的电路组合而成的。


% \subsection{两级运放的频率特性}
由于构成两级运算放大器的差分放大器和反相放大器我们都已经分析的相当透彻了,这里小信号电路不需要从晶体管画起,如\xref{fig:两级运算放大器的小信号电路}所示,我们可以将每一级放大器视为一个具有一定的跨导$G_{m}$、输出电阻$R$、输出电容$C$的黑盒。这里要说明两点,第一,尽管研究单级放大器的时候,更常用的小信号参数是增益$A_v$和输出电阻$R_{out}$,但是用跨导$G_m$和输出电阻$R_{out}$表示其实可以更好凸显“输入电压先被放大管转换为电流再在负载管上转换回电压输出”的直观分析思路,且跨导$G_m$的表达式可以很容易的从$A_v=-G_m R_{out}$截取到。第二,这里的建模只考虑了每一级放大器输出节点处的电容,忽略了其他内部电容。这是个合理的近似,因为参考之前的经验,在分析频率特性时我们会考虑想当多的电容,但是,到了真正设计的时候我们只会考虑输出电容,因为负载电容远大于寄生电容,只有输出节点的电容是较重要的。
\begin{Figure}[两级运放的小信号电路]
    \includegraphics[scale=0.8]{build/Chapter05B_03.fig.pdf}
\end{Figure}

这里$G_{m1},R_{1},G_{m2},R_{2}$的表达式参照\xref{fml:电流镜负载差分放大器--右侧输出--差模增益}、\xref{fml:电流镜负载差分放大器--右侧输出--输出电阻}、\xref{fml:电流源负载反相放大器--电压增益}、\xref{fml:电流源负载反相放大器--输出电阻}确定为
\begin{BoxFormula}[两级运放--等效跨导和输出电阻]
    两级运放,两级放大器的等效跨导和输出电阻为
    \begin{Gather}
        G_{m1}=g_{m2}\\
        G_{m2}=g_{m6}\\
        R_1=(g_{ds2}+g_{ds4})^{-1}\\
        R_2=(g_{ds6}+g_{ds7})^{-1}
    \end{Gather}
\end{BoxFormula}

现在求$A_v(s)=v_{out}/v_{in}$,就\xref{fig:两级运放的小信号电路}的$v_{c},v_{out}$节点列出方程
\begin{Gather}
    G_{m1}v_{in}+sC_1v_v+R_1^{-1}v_c=0\\
    G_{m2}v_c+sC_2v_{out}+R_2^{-1}v_{out}=0
\end{Gather}
我们可以解得
\begin{Equation}
    A_{v}(s)=\frac{G_{m1}G_{m2}R_1R_2}{(1+sR_1C_1)(1+sR_2C_2)}
\end{Equation}
这个结果是显然的:运算放大器的增益$A_v(s)$是组成其的两个单级放大器的增益之积。同时注意到,两级运算放大器具有两个极点$\omega_{p1},\omega_{p2}$,分别由第一级和第二级的输出节点产生。

% 第一级放大器和第二级放大器的输出节点处的$R_1,C_1$和$R_2,C_2$分别提供了一个极点。

% 这里分子是低频增益$A_v=G_{m1}G_{m2}R_1R_2$,从分母上很容易判读出两个极点
% \begin{Equation}
%     \omega_{p1}=-R_1^{-1}C_1^{-1}\qquad \omega_{p2}=-R_2^{-1}C_2^{-1}
% \end{Equation}

\begin{BoxFormula}[两级运放--增益]
    两级运放,低频增益为
    \begin{Equation}
        A_v=G_{m1}G_{m2}R_1R_2
    \end{Equation}
\end{BoxFormula}

\begin{BoxFormula}[两级运放--零极点]
    两级运放,零极点为
    \begin{Gather}
        \omega_{p1}=-R_1^{-1}C_1^{-1}\\
        \omega_{p2}=-R_2^{-1}C_2^{-1}
    \end{Gather}
\end{BoxFormula}
一切看起来都很简单,不是吗?从结构上看,运算放大器不过就是两个单级放大器的级联,从性质上看,运算放大器包括增益和零极点在内的各种特性似乎都可以由构成其的单级放大器得到。那关于运算放大器到底还有什么新东西要研究?请看\xref{sec:两级运算放大器的补偿原理}有关“补偿”的讨论。

% 实际上,运算放大器真正困难的地方不在这里,