\section{MOS二极管}
本节将介绍MOS二极管。MOS二极管是指将MOS管的栅和漏短接后构成的二端器件,将其称为“二极管”是因为其特性与通常的PN结二极管有相似之处,都具备整流特性,不过这种相似性并不是研究的重点。MOS二极管有NMOS和PMOS两种类型,如\xref{fig:MOS二极管}所示。

\begin{Figure}[MOS二极管]
    \begin{FigureSub}[NMOS二极管]
        \qquad
        \includegraphics[scale=0.8]{build/Chapter03B_02.fig.pdf}
        \qquad
    \end{FigureSub}
    \begin{FigureSub}[PMOS二极管]
        \qquad
        \includegraphics[scale=0.8]{build/Chapter03B_03.fig.pdf}
        \qquad
    \end{FigureSub}
\end{Figure}
以下我们均以NMOS二极管为例进行研究。

第一步,分析MOS二极管的大信号特征,由于$v_{DS}=v_{GS}$,显然恒有$v_{DS}\geq v_{GS}-V_T$,故采用二极管接法的MOS管始终工作在饱和区。令饱和区方程中$i_D=i_{OUT}$和$v_{GS}=v_{OUT}$,得
\begin{BoxFormula}[MOS二极管的大信号特性]
    MOS二极管的大信号特性为
    \begin{Equation}
        i_{OUT}=\frac{\beta}{2}(v_{OUT}-V_T)^2(1+\lambda v_{OUT})
    \end{Equation}
\end{BoxFormula}
\xref{fig:MOS二极管的大信号特性}展示了MOS二极管的大信号特性
\begin{Figure}[MOS二极管的大信号特性]
    \includegraphics[scale=0.6]{build/Chapter03B_01_0.fig.pdf}
\end{Figure}

第二步,分析MOS二极管的小信号特征,这主要就是其输出电阻$r_{out}=v_{out}/i_{out}$,由于源接地故$g_{bs}$可忽略,由于栅漏短接故跨导$g_{m}$现在实质上也和$g_{ds}$一样成为了电导,如\xref{fig:MOS二极管的小信号电路}

\begin{Figure}[MOS二极管的小信号电路]
    \includegraphics[scale=0.8]{build/Chapter03B_04.fig.pdf}
\end{Figure}
列出方程\setpeq{MOS二极管小信号}
\begin{Equation}&[1]
    i_{out}=g_mv_{gs}+g_{ds}v_{ds}
\end{Equation}
这里端电压可以表示为
\begin{Equation}&[2]
    \begin{pmatrix}
        v_{gs}\\
        v_{bs}\\
        v_{ds}
    \end{pmatrix}=
    \begin{pmatrix}
        v_{out}\\
        0\\
        v_{out}
    \end{pmatrix}
\end{Equation}
将\xrefpeq{2}代入\xrefpeq{1}
\begin{Equation}
    i_{out}=g_mv_{out}+g_{ds}v_{out}
\end{Equation}
整理得到
\begin{Equation}
    i_{out}=(g_m+g_{ds})v_{out}
\end{Equation}
因此
\begin{Equation}
    R_{out}=\frac{v_{out}}{i_{out}}=(g_m+g_{ds})^{-1}
\end{Equation}
这一结果最为精辟的展示了MOS二极管“栅漏短接”使“跨导变为电导”的特性。
\begin{BoxFormula}[MOS二极管的输出电阻]
    MOS二极管的输出电阻为
    \begin{Equation}
        R_{out}=(g_m+g_{ds})^{-1}
    \end{Equation}
\end{BoxFormula}